\chapter{Podsumowanie}
\label{chap:podsumowanie}
Celem pracy było zaprojektowanie i zaimplementowanie systemu do audio-wizualnego alarmowania w sytuacji wykrycia obiektów z podzbioru dostępnych klas ustawionych przez użytkownika. Ponadto należało przeprowadzić badania szybkości systemu oraz badania dokładności detekcji obiektów dla wykorzystanego modelu sztucznej inteligencji.
Cel systemu oraz postawione wymagania udało się zrealizować.


Testy szybkości, przeprowadzone na zasobach sprzętowych posiadanych przez przeciętnego użytkownika, wykazały zadowalające wyniki oraz przybliżoną wydajność dla różnych ustawień rozdzielczości kamery. Testy możliwości detekcji modelu YOLOv8n wykazały realną możliwość użycia systemu w warunkach oświetlenia uznanych za wystarczająco dobre. System może być użyty przez osobę, która nie posiada zasobów na zakup specjalistycznych rozwiązań.
W przypadku wspomnianych warunków oświetlenia badania modelu wykazały bardzo dobrą skuteczność w klasyfikacji ludzi w ruchu, a co za tym idzie -- w alarmowaniu dźwiękowym dla scen z niewielką liczbą obecnych obiektów, w sytuacji dobrze oświetlonych, widocznych konturów człowieka. Przykładem użycia systemu w takich warunkach oświetlenia mógłby scenariusz alarmowania w sytuacji, gdy obiekt przechodzi przez wejście do pomieszczenia -- scenariusz podobny do scenariusza testowego.  


Ze względu na brak bardziej zaawansowanych zasobów sprzetowych nie było możliwe przeprowadzenie badań dla różnych konfiguracji komponentów, a następnie porównanie uzyskanych wyników. Jest to natomiast punkt, od którego interesująco byłoby rozpocząć kolejne badania, kładąc szczególny nacisk na wykorzystanie kamery o możliwości dostarczania większej liczby klatek na sekundę. Kamera o większym FPS mogłaby przyśpieszyć działanie modułu obługi kamery względem modułu detekcji obiektów, co mogłoby wpłynąć na charakterystykę wyników testów. Możnaby również powielić przeprowadzone badania -- modelu oraz szybkości systemu -- lecz dla innych rozmiarów modelu YOLOv8, np. YOLOv8s.  

%\show\chapter
%\show\section
%\show\subsection

%\showthe\secindent
%\showthe\beforesecskip
%\showthe\aftersecskip
%\showthe\secheadstyle
%\showthe\subsecindent
%\showthe\beforesubsecskip
%\showthe\aftersubsecskip
%\showthe\subseccheadstyle
%\showthe\parskip
