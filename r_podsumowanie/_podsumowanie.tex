\chapter{Podsumowanie}
\label{chap:podsumowanie}
Celem pracy było zaprojektowanie oraz zaimplementowanie systemu do audio-wizualnego alarmowania użytkownika w sytuacji wykrycia obiektów z podzbioru dostępnych klas ustawionych przez użytkownika. Ponadto należało przeprowadzić badania szybkości systemu oraz badania dokładności problemu detekcji obiektów dla wykorzystanego modelu sztucznej inteligencji.
Cel systemu oraz postawione wymagania udało się zrealizować.


Testy szybkości przeprowadzone na zasobach sprzętowych posiadanych przez przeciętnego użytkownika wykazały zadowalające wyniki oraz przybliżoną wydajność dla różnych ustawień rozdzielczości kamery. Badania wykazały więc realną możliwość użycia systemu w warunkach oświetlenia uznanych za wystarczająco dobre. System może być użyty przez osobę, która nie posiada zasobów na zakup specjalistycznego systemu, np. w celu monitoringu swojej posiadłości.
W przypadku w.w. warunków oświetlenia, badania modelu wykazały bardzo dobrą skuteczność klasyfikacji ludzi w ruchu, a co za tym idzie alarmowania dźwiękowego dla scen z pojedyńczą liczbą obecnych obiektów, w sytuacji dobrze oświetlonych, widocznych konturów człowieka. Przykładem użycia systemu w w.w. warunkach oświetlenia mógłby być scenariusz do alarmowania przejścia obiektu przez wejście do pomiesczenia -- scenariusz podoby do scenariusza testowego.  


Ze względu na brak zasobów sprzętowych o parametrach odpowiadających warunkom profesjonalnego użycia tego typu systemów nie było możliwości porówania niniejszego oprogramowania do istniejących rozwiązań. Jest to natomiast punkt, od którego interesująco byłoby rozpocząć kolejne badania, kładąc szczególny nacisk na wykorzystanie kamery o możliwości dostarczania większej liczby klatek na sekundę, gdyż użycie takiej technologii mogłoby przyśpieszyć działanie wolniejszych modułów systemu i zmienić charakterystykę wyników. Możnaby również powielić przeprowadzone badania -- modelu oraz szybkości systemu -- lecz dla innych rozmiarów modelu YOLOv8.  

%\show\chapter
%\show\section
%\show\subsection

%\showthe\secindent
%\showthe\beforesecskip
%\showthe\aftersecskip
%\showthe\secheadstyle
%\showthe\subsecindent
%\showthe\beforesubsecskip
%\showthe\aftersubsecskip
%\showthe\subseccheadstyle
%\showthe\parskip
