\pdfbookmark[0]{Skróty}{skroty.1}% 
%%\phantomsection
%%\addcontentsline{toc}{chapter}{Skróty}
\chapter*{Wykaz skrótów i pojęć}
\noindent\vspace{-\topsep-\partopsep-\parsep}
\section*{Skróty}
\begin{description}
\label{sec:skroty}
      \item [CNN] (ang.\ \emph{convolutional neural network}) -- konwolcyjna sieć neuronowa

  \item [GPU] (ang.\ \emph{graphics processing unit}) -- procesor graficzny
  
  \item [CPU] (ang.\ \emph{central processing unit}) -- procesor

  \item [GUI] (ang.\ \emph{Graphical User Interface}) -- graficzny interfejs użytkownika

  \item [FPS] (ang.\ \emph{frames per second}) -- liczba klatek na sekundę   
  
  \item [TP] (ang.\ \emph{true positive}) -- wynik prawdziwy pozytywny

  \item [TN] (ang.\ \emph{true negative}) -- wynik prawdziwy negatywny

  \item [FP] (ang.\ \emph{false positive}) -- wynik fałszywy pozytywny

  \item [FN] (ang.\ \emph{false negative}) -- wynik fałszywy negatywny

\end{description}

\section*{Pojęcia}
\begin{description}
  \label{sec:pojecia}
    \item [próg ufności] (oryg. \emph{conf})  -- parametr inferencji w wykorzystanym interfejsie programistycznym YOLOv8. Filtruje wyniki detekcji i odrzuca te poniżej progu ufności.
  
    \item [AUC] (ang.\ \emph{area under curve}) -- pole pod wykresem krzywej np. krzywej ROC.
  
    \item [ROC] (ang.\ \emph{receiver operating characteristic}) -- sposób wizualizacji skuteczności klasyfikacji binarnej.
  \end{description}
