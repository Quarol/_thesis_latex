\chapter{Testy możliwości detekcji obiektów}
Rozdział ten opisuje autorskie testy dokładności detekcji obiektów w zależności od różnych czynników. Testowanym zadaniem jest dokładność klasyfikacji obiektów. Nie zaimplementowano testów dokładności dla lokalizacji przestrzennej, ponieważ nie jest ona priorytetem w niniejszym systemie. Uznano, iż jakościowe stwierdzenie możliwości lokalizacji w fazie tworzenia systemu jest wystarczające, a efekt uznano za zadowalający. 

Zaimplementowane dwa testy skuteczności klasyfikacji to:
\begin{enumerate}
    \item Test pojedyńczej klasy obiektu przy użyciu różnych metryk na przykładzie klasy \emph{człowiek} (ang. \emph{person}). 
    \item  Test poziomu generalizacji modelu w scenariuszu obecności różnych wariantów obiektu wchodzących w skład tej samej klasy. Test na przykładzie klasy \emph{krzesło} (ang. \emph{chair}) i użytych wariantów: krzesło kuchenne, krzesło gamingowe. 
\end{enumerate}
    

Rozdział ten wpierw porusza kwestie współdzielone przez oba testy takie jak źródło wideo i pobranie z niego wyników użytych do analizy w posczególnym scenariuszu. Następnie opisane zostaną już koknkretne scenariusze.
 
Uwaga dotycząca zamieszconych rysunków: w rysunkach przedstawiających człowieka twarz została zamazana.


\section{Przygotowanie danych do testów}
\subsection{Żródło wideo}
\label{sec:zrodlo_wideo}
Źródło wideo to nagrane w domowych warunkach, autorskie, krótkie filmy. Filmy nagrano na kamerze o rozdzielczości 640x480 pikseli. Obiektyw kamery jest statycznie usytuowany w tej samej lokalizacji dla każdego filmu. Obiektyw jest skierowany na wejście do pomieszczenia, w którym kamera się znajduje. Wygląd pomieszczenia jest przedstawiony na ryskunku \ref{fig:test-dokladnosc-scena}.

\begin{figure}[H]
    \centering
    \includegraphics[width=\linewidth]{r_test_dokładności/vid_pics/1_1.jpg}
    \caption{Pomieszczenie, w którym nagrano filmy.}
    \label{fig:test-dokladnosc-scena}
\end{figure}

\begin{figure}[H]
    \centering
    \begin{minipage}{0.32\textwidth}
        \centering
        \includegraphics[width=\linewidth]{r_test_dokładności/vid_pics/1_2.jpg}
        \caption{Klatka filmu z człowiekiem.}
    \end{minipage}\hfill
    \begin{minipage}{0.32\textwidth}
        \centering
        \includegraphics[width=\linewidth]{r_test_dokładności/vid_pics/1c_2.jpg}
        \caption{Klatka filmu z człowiekiem i krzesłem.}
    \end{minipage}\hfill
    \begin{minipage}{0.32\textwidth}
        \centering
        \includegraphics[width=\linewidth]{r_test_dokładności/vid_pics/1g_2.jpg}
        \caption{Klatka filmu z człowiekiem i fotelem.}
    \end{minipage}
    \caption{Prezentacja użytych obiektów.}
    \label{fig:all_objects}
\end{figure}
Filmy można podzielić na trzy kategorie ze względu na różne obecne w nim obiekty (wizualizacja objektów: rysunek \ref{fig:all_objects}):
\begin{itemize}
    \item człowiek
    \item człowiek, krzesło
    \item człowiek, fotel
\end{itemize}
Krzesło oraz fotel to obiekty, z perspektywy filmów, statyczne --- przez całą długość filmu, na których występują, są one umieszczone w jednej lokalizacji, a inne obiekty nie mają z nimi fizycznej integracji.
Natomiast człowiek jest obiektem ruchomym. Jest on nieobecny przez pierwszą część każdego filmu. W drugiej części przechodzi on przez próg pomieszczenia, przybliża się do pewnego punktu w obiektywie i na koniec filmu robi jeden pełen obrót w okół własnej osi. 

Należy podkreślić ważne kwestie dotyczące nagranych filmów:
\begin{itemize}
    \item Obecne obiekty nie wchodzą ze sobą w integrację --- człowiek podczas ruchu, nie zasłania konturów krzesła oraz fotela.
    \item Generacja danych do testów nie odbywała się na każdej klatcę nagranego filmu. Użyto tylko klatek przedstawiających pełny kontur obiektów. Pełny kontur człowieka pojawia się w momencie przekroczenia progu wejścia do pomieszczenia. Użyto więc dwóch sekcji każdego filmu:
    \begin{itemize}
        \item Sekcja gdy człowiek był całkowicie nieobecny.
        \item Sekcja od moment przekroczenia progu.
    \end{itemize}
\end{itemize}

Dla każdej kategorii filmu nagrano cztery warianty z różnym poziomem oświetlenia w pomieszczeniu, co daje łącznie dwanaście plików wideo do analizy. Wizualne różnice między poziomami przedstawiono na rysunku \ref{fig:person_grid} (człowiek), rysunku \ref{fig:chair_grid} (człowiek, krzesło) i rysunku \ref{fig:game_grid} (człowiek, fotel). 
Dla każdego filmu, korzystając z modelu przestrzeni barw HSV, obliczono średnią jasność oraz nasycenie filmu. Wartości te zaprezentowano w tabeli \ref{tab:saturation-value-table}. Obliczenia przeprowadzono poprzez wyznaczenie średniej wartości jasności oraz nasycenia dla każdej klatki obrazu na podstawie wszystkich pikseli, a następnie wyznaczając średnią z uzyskanych wartości dla wszystkich klatek w filmie.

\begin{table}[H]
\centering
\caption{Jasność i nasycenie dla każdego nagranego filmu.}
\begin{tabular}{|c|c|c|c|}
\hline
Nr sceny:          & Obecne obiekty    & Jasność & Nasycenie \\ \hline
\multirow{3}{*}{1} & człowiek          & 152.73  & 107.3     \\ \cline{2-4} 
                   & człowiek, krzesło & 149.92  & 119.07    \\ \cline{2-4} 
                   & człowiek, fotel   & 152.68  & 111.47    \\ \hline
\multirow{3}{*}{2} & człowiek          & 134.64  & 93.48     \\ \cline{2-4} 
                   & człowiek, krzesło & 139.14  & 90.71     \\ \cline{2-4} 
                   & człowiek, fotel   & 133.77  & 83.29     \\ \hline
\multirow{3}{*}{3} & człowiek          & 41.59   & 123.48    \\ \cline{2-4} 
                   & człowiek, krzesło & 38.91   & 132.31    \\ \cline{2-4} 
                   & człowiek, fotel   & 38.12   & 124.31    \\ \hline
\multirow{3}{*}{4} & człowiek          & 25.47   & 90.92     \\ \cline{2-4} 
                   & człowiek, krzesło & 25.3    & 100.8     \\ \cline{2-4} 
                   & człowiek, fotel   & 24.88   & 108.5     \\ \hline
\end{tabular}
\label{tab:saturation-value-table}
\end{table}
\begin{figure}[H]
    \centering
    \begin{minipage}{0.45\textwidth}
        \centering
        \includegraphics[width=\linewidth]{r_test_dokładności/vid_pics/1_2.jpg}
    \end{minipage}\hfill
    \begin{minipage}{0.45\textwidth}
        \centering
        \includegraphics[width=\linewidth]{r_test_dokładności/vid_pics/2_2.jpg}
    \end{minipage}
    \vskip\baselineskip
    \begin{minipage}{0.45\textwidth}
        \centering
        \includegraphics[width=\linewidth]{r_test_dokładności/vid_pics/3_2.jpg}
    \end{minipage}\hfill
    \begin{minipage}{0.45\textwidth}
        \centering
        \includegraphics[width=\linewidth]{r_test_dokładności/vid_pics/4_2.jpg}
    \end{minipage}
    \caption{Poziomy oświetlenia dla filmów z obecnym człowiekiem}
    \label{fig:person_grid}
\end{figure}
\begin{figure}[H]
    \centering
    \begin{minipage}{0.45\textwidth}
        \centering
        \includegraphics[width=\linewidth]{r_test_dokładności/vid_pics/1c_2.jpg}
    \end{minipage}\hfill
    \begin{minipage}{0.45\textwidth}
        \centering
        \includegraphics[width=\linewidth]{r_test_dokładności/vid_pics/2c_2.jpg}
    \end{minipage}
    \vskip\baselineskip
    \begin{minipage}{0.45\textwidth}
        \centering
        \includegraphics[width=\linewidth]{r_test_dokładności/vid_pics/3c_2.jpg}
    \end{minipage}\hfill
    \begin{minipage}{0.45\textwidth}
        \centering
        \includegraphics[width=\linewidth]{r_test_dokładności/vid_pics/4c_2.jpg}
    \end{minipage}
    \caption{Poziomy oświetlenia dla filmów z obecnym człowiekiem i krzesłem}
    \label{fig:chair_grid}
\end{figure}
\begin{figure}[H]
    \centering
    \begin{minipage}{0.45\textwidth}
        \centering
        \includegraphics[width=\linewidth]{r_test_dokładności/vid_pics/1g_2.png}
    \end{minipage}\hfill
    \begin{minipage}{0.45\textwidth}
        \centering
        \includegraphics[width=\linewidth]{r_test_dokładności/vid_pics/2g_2.png}
    \end{minipage}
    \vskip\baselineskip
    \begin{minipage}{0.45\textwidth}
        \centering
        \includegraphics[width=\linewidth]{r_test_dokładności/vid_pics/3g_2.png}
    \end{minipage}\hfill
    \begin{minipage}{0.45\textwidth}
        \centering
        \includegraphics[width=\linewidth]{r_test_dokładności/vid_pics/4g_2.png}
    \end{minipage}
    \caption{Poziomy oświetlenia dla filmów z obecnym człowiekiem i fotelem}
    \label{fig:game_grid}
\end{figure}







\subsection{Metryki i dane wynikowe używane w testach}
Na podstawie opisanych dwunastu plików wideo wygenerownao dane do analizy w testach dla każdego z nich. Dane te to podstawowe metryki używane w klasyfikacji binarnej. Są to: wynik prawdziwy pozytywny (TP), wynik prawdziwy negatywny (TN), wynik fałszywy pozytywny (FP), wynik fałszywy negatywny (FN). 
Zaprojektowane testy sprawdzają tylko jedną klasę na test --- np. dla ewaluacji klasyfikacji klasy \emph{człowiek}, ignorowane są wykrycia innych klas. Co więcej, w niniejszym systemie nacisk kładziony jest na alarmowanie użytkownika o obecności obiektu, dlatego też uznano, iż więcej niż jedna detekcja klasy będzie przypisana do tej samej metryki co dotekcja pojedyńczego obiektu. 

Biorąc to pod uwagę, definicję metryk skonstruowano następująco: \\ \\ \noindent
Znaczenie skrótów z tabeli \ref{tab:saturation-value-table}: \\
$\emptyset$ -- brak wystąpienia obiektu wykrywanej klasy \\
x -- pojedyńcze wystąpienie obiektu wykrywanej klasy \\
n*x -- wielokrotne wystąpienie obiektu wykrywanej klasy
\begin{table}[H]
    \centering
    \caption{Definicja metryk generowanych podczas testów detekcji obiektów.}
    \begin{tabular}{|c|c|c|c|}
    \hline

    Metryka & Obiekty na nagraniu & Obiekty wykryte przez detektor & Opis \\ \hline



    \multirow{2}{*}{TP} & \multirow{2}{*}{x} & x & \multirow{2}{*}{Detektor wykrył jeden bądź wiele obiektów klasy $x$ gdy obiekt $x$ był obecny.} \\ \cline{3-3}

    & & n*x & \\ \hline


    TN & $\emptyset$ & $\emptyset$ & Detektor nie wykrył żadnego obiektu klasy $x$ gdy obiekt $x$ nie był obecny. \\ \hline
    


    \multirow{2}{*}{FP} & \multirow{2}{*}{$\emptyset$} & x & \multirow{2}{*}{Detektor wykrył jeden bądź wiele obiektów klasy $x$ gdy obiekt $x$ nie był obecny.} \\ \cline{3-3}

    & & n*x & \\ \hline


    
    FN & x & $\emptyset$ & Detektor nie wykrył żadnego obiektu klasy $x$ gdy obiekt $x$ był obecny. \\ \hline

    \end{tabular}
\end{table}

Generacja metryk wynikowych odbywa się w następująco:
\begin{enumerate}
    \item Wybierany jest film.
    \item Dla filmu metryki generowane są dla każdej badanej wartości progu ufności.
    \item Dla progu ufności analizowana jest każda klatka spełniająca wymagania opisane w rozdziale \ref{sec:zrodlo_wideo}. Dana klatka jest kategoryzowana do jednej z metryk.
    \item Metryka wynikowa to liczba klatek przypisanych do tej metryki.
\end{enumerate} 
Przykład wygenerowanych wyników prezentuje tabela \ref{tab:example-generated}.
\begin{table}[H]
    \centering
    \caption{Struktura wygenerowanych wyników na przykładzie częściowych wyników dla filmu z obecnym człowiekiem dla poziomu oświetlania nr 1. Wartości metryk to liczba klatek filmu przypisana do każdej z nich.}
    \begin{tabular}{ccccc}
    \hline
    \multicolumn{1}{|c|}{Próg ufności} & \multicolumn{1}{c|}{TP}                    & \multicolumn{1}{c|}{TN}                    & \multicolumn{1}{c|}{FP}                    & \multicolumn{1}{c|}{FN}                    \\ \hline
    \multicolumn{1}{|c|}{0}                                 & \multicolumn{1}{c|}{100}                   & \multicolumn{1}{c|}{0}                     & \multicolumn{1}{c|}{122}                   & \multicolumn{1}{c|}{0}                     \\ \hline
    \multicolumn{1}{|c|}{0.01}                              & \multicolumn{1}{c|}{100}                   & \multicolumn{1}{c|}{49}                    & \multicolumn{1}{c|}{73}                    & \multicolumn{1}{c|}{0}                     \\ \hline
    \multicolumn{1}{|c|}{0.02}                              & \multicolumn{1}{c|}{100}                   & \multicolumn{1}{c|}{104}                   & \multicolumn{1}{c|}{18}                    & \multicolumn{1}{c|}{0}                     \\ \hline
    \multicolumn{1}{|c|}{0.03}                              & \multicolumn{1}{c|}{100}                   & \multicolumn{1}{c|}{120}                   & \multicolumn{1}{c|}{2}                     & \multicolumn{1}{c|}{0}                     \\ \hline
    \multicolumn{1}{|c|}{0.04}                              & \multicolumn{1}{c|}{100}                   & \multicolumn{1}{c|}{122}                   & \multicolumn{1}{c|}{0}                     & \multicolumn{1}{c|}{0}                     \\ \hline
    \multicolumn{5}{c}{$\bullet$ $\bullet$   $\bullet$ $\bullet$ $\bullet$ $\bullet$ $\bullet$ $\bullet$ $\bullet$   $\bullet$ $\bullet$ $\bullet$ $\bullet$ $\bullet$ $\bullet$ $\bullet$   $\bullet$ $\bullet$ $\bullet$ $\bullet$ $\bullet$} \\ \hline
    \multicolumn{1}{|c|}{0.96}                              & \multicolumn{1}{c|}{0}                     & \multicolumn{1}{c|}{122}                   & \multicolumn{1}{c|}{0}                     & \multicolumn{1}{c|}{100}                   \\ \hline
    \multicolumn{1}{|c|}{0.97}                              & \multicolumn{1}{c|}{0}                     & \multicolumn{1}{c|}{122}                   & \multicolumn{1}{c|}{0}                     & \multicolumn{1}{c|}{100}                   \\ \hline
    \multicolumn{1}{|c|}{0.98}                              & \multicolumn{1}{c|}{0}                     & \multicolumn{1}{c|}{122}                   & \multicolumn{1}{c|}{0}                     & \multicolumn{1}{c|}{100}                   \\ \hline
    \multicolumn{1}{|c|}{0.99}                              & \multicolumn{1}{c|}{0}                     & \multicolumn{1}{c|}{122}                   & \multicolumn{1}{c|}{0}                     & \multicolumn{1}{c|}{100}                   \\ \hline
    \multicolumn{1}{|c|}{1}                                 & \multicolumn{1}{c|}{0}                     & \multicolumn{1}{c|}{122}                   & \multicolumn{1}{c|}{0}                     & \multicolumn{1}{c|}{100}                   \\ \hline
    \end{tabular}
    \label{tab:example-generated}
    \end{table}


\section{Test 1}
Niniejszy podrozdział przedstawia różne badania, których celem jest ewaluacja klasyfikacji YOLOv8n na przykładzie klasy człowiek. Model zbadano pod kątem wpływu jak manipulacja poziomem oświetlenia oraz progu ufności przekłada się na jego wyniki. Dla każdego zmierzonego progu ufności wyznaczane będą metryki potrzebne do zobrazowania wniosków. Będą one liczone na podstawie poprzednio uzyskanych metryk --- TP, TN, FP i FN. Do testu wykorzystano cztery filmy. Każdy z nich przedstawia wyłącznie człowieka. Wartości jasności i nasycenia przedstawiono w tabeli \ref{tab:saturacja-jasnosc-czlowiek}. Wygląd nagranego pomieszczenia dla różnych poziomów oświetlenia ukazano na rysunku \ref{fig:person_grid}. 
\begin{table}[H]
\centering
\caption{Jasność i nasycenie dla wszystkich filmów.}
\begin{tabular}{|c|c|c|c|}
\hline
Poziom oświetlenia  & Obecne obiekty & Jasność & Nasycenie \\ \hline
1        & człowiek       & 152.73  & 107.3     \\ \hline
2        & człowiek       & 134.64  & 93.48     \\ \hline
3        & człowiek       & 41.59   & 123.48    \\ \hline
4        & człowiek       & 25.47   & 90.92     \\ \hline
\end{tabular}

\label{tab:saturacja-jasnosc-czlowiek}
\end{table}








\subsection{Zachowanie modelu dla różnych poziomów oświetlenia}
\label{sec:test-AUC}
W sekcji tej oceniono osiągi modelu dla każdego poziomu oświetlania, a następnie porównano je ze sobą. Do wizualizacji wyników posłużono się krzywą ROC, zaś do oceny rezultatów dla poziomu oświetlenia -- wartością AUC.

Krzywa ROC (ang. \emph{Receiver operating characteristic}) to graficzna reprezentacja wykorzystywana do ewaluacji binarnej klasyfikacji modelu. Jest to zbiór połączonych ze sobą punktów współrzędnych $(x, y)$. Liczba narysowanych punktów odpowiada liczbie przetestowanych progów ufności. Współrzędna $x$ to metryka FPR, a współrzędna $y$ -- TPR. Przykład wyznaczonych punktów pokazuje tabela \ref{tab:wyznaczanie_ROC}. 
    \begin{table}[H]
        \centering
    \caption{Przykład wyznaczania punktów krzywej ROC}

        \begin{tabular}{|c|c|c|c|c|c|c|c|}
        \hline
        Próg   ufności & TP & TN & FP & FN & FPR  & TPR  & Punkt na   wykresie \\ \hline
        0              & 60 & 0  & 26 & 0  & 1    & 1    & (1, 1)              \\ \hline
        0.01           & 55 & 0  & 26 & 5  & 1    & 0.92 & (1, 0.92)           \\ \hline
        0.02           & 38 & 3  & 23 & 22 & 0.88 & 0.63 & (0.88, 0.63)        \\ \hline
        0.03           & 22 & 13 & 13 & 38 & 0.5  & 0.37 & (0.5, 0.37)         \\ \hline
        0.04           & 18 & 16 & 10 & 42 & 0.38 & 0.3  & (0.38, 0.3)         \\ \hline
        0.05           & 11 & 19 & 7  & 49 & 0.27 & 0.18 & (0.27, 0.18)        \\ \hline
        0.06           & 8  & 21 & 5  & 52 & 0.19 & 0.13 & (0.19, 0.13)        \\ \hline
        0.07           & 6  & 22 & 4  & 54 & 0.15 & 0.1  & (0.15, 0.1)         \\ \hline
        0.08           & 5  & 22 & 4  & 55 & 0.15 & 0.08 & (0.15, 0.08)        \\ \hline
        0.09           & 3  & 23 & 3  & 57 & 0.12 & 0.05 & (0.12, 0.05)        \\ \hline
        0.1            & 3  & 23 & 3  & 57 & 0.12 & 0.05 & (0.12, 0.05)        \\ \hline
        0.11           & 3  & 24 & 2  & 57 & 0.08 & 0.05 & (0.08, 0.05)        \\ \hline
        0.12           & 3  & 25 & 1  & 57 & 0.04 & 0.05 & (0.04, 0.05)        \\ \hline
        0.13           & 3  & 25 & 1  & 57 & 0.04 & 0.05 & (0.04, 0.05)        \\ \hline
        0.14           & 3  & 25 & 1  & 57 & 0.04 & 0.05 & (0.04, 0.05)        \\ \hline
        0.15           & 3  & 25 & 1  & 57 & 0.04 & 0.05 & (0.04, 0.05)        \\ \hline
        0.16           & 3  & 25 & 1  & 57 & 0.04 & 0.05 & (0.04, 0.05)        \\ \hline
        0.17           & 3  & 25 & 1  & 57 & 0.04 & 0.05 & (0.04, 0.05)        \\ \hline
        0.18           & 2  & 26 & 0  & 58 & 0    & 0.03 & (0, 0.03)           \\ \hline
        0.19           & 2  & 26 & 0  & 58 & 0    & 0.03 & (0, 0.03)           \\ \hline
        0.2            & 2  & 26 & 0  & 58 & 0    & 0.03 & (0, 0.03)           \\ \hline
        \end{tabular}
    \label{tab:wyznaczanie_ROC}

        \end{table}

TPR (ang. True Positive Rate) to metryka, która demonstruje, jak dobrze model poradził sobie w sytuacji, kiedy obiekt był obecny na filmie.Innymi słowy, pokazuje, ile dobrych decyzji (TP) podjął model względem wszystkich podjętych decyzji w sytuacji obecności obiektu. Dla zbioru klatek z obecnym obiektem, TPR jest stosunkiem liczby klatek, w których obiekt został wykryty (TP), do całkowitej liczby klatek z tego zbioru (TP + FN). Metryka ta jest obliczana według wzoru \ref{eq:TPR}. Z perspektywy całego systemu TPR opisuje skuteczność alarmowania użytkownika w sytuacji pojawienia się obiektu -- czułość modelu. Im większa wartość TPR, tym większa czułość modelu.

\begin{equation}
    TPR = \frac{TP}{TP + FN}
    \label{eq:TPR}
\end{equation}

FPR (ang \emph{false positive rate}) opisuje sytuację odwrotną do TPR --- analizowane są klatki kiedy obiekt nie był obecny. Pokazuje ile złych decyzji (FP) zostało podjętych względem wszystkich podjętych decyzji w sytuacji nieobecnośći obiektu. Dla zbioru klatek z nieobecnym obiektem, FPR to stosunek liczby klatek kiedy model wykrył obiekt (FP) do całkowitej liczby klatek z tego zbioru (FP + TN). Metryka ta jest obliczana według wzoru \ref{eq:FPR}. Z perspektywy całego systemu FPR opisuję skuteczność unikania generacji fałszywych alarmów -- ostrożność modelu. Im niższa wartość FPR, tym mniej fałszywych alarmów.


\begin{equation}
    FPR = \frac{FP}{FP + TN}
    \label{eq:FPR}
\end{equation}

Podsumowując, ROC ilustruje balans między czułością a ostrożnością modelu. Do interpretacji krzywej używa się wartości skalarnej AUC. AUC (ang. \emph{area under curve}) jest to pole pod wykresem krzywej ROC. Jest to wartość z przedziału [0, 1]. Im większa wartość, tym lepszy wynik modelu. AUC równa 1 oznacza, że model idealnie wskazuje kiedy obiekt się pojawił oraz nigdy nie generuje fałszywych wyników gdy obiekt jest nieobecny. Wartość 0.5 określana jest jako wynik uzyskany przez tzw. losowego klasyfikatora (ang. \emph{random classifier}). Losowy klasyfikator definiuje się jako klasyfikator losowo generujący wykrycia bądź nie (zakładając, że prawdopodobieństwo wykrycia to 50\%). Obrazuje się go prostą, przekątną linią, biegnącą od lewego-dolnego rogu wykresu do prawego-górnego. Modele uzyskujące AUC mniejsze niż 0.5 są uznawane za nieskuteczne. 
Wykresy dla kolejnych poziomów oświetlenia przedstawiono na rysunkach \ref{fig:ROC-1}, \ref{fig:ROC-2}, \ref{fig:ROC-3}, \ref{fig:ROC-4}. Porównanie wartości AUC wszystkich poziomów oświetlenia ukazane jest na wykresie na rysunku \ref{fig:AUC}.



\begin{figure}[H]
    \centering
    \includegraphics[width=\linewidth]{r_test_dokładności/AUC_charts/1.png}
    \caption{Wykres krzywej ROC i wartość AUC dla poziomu oświetlenia nr 1.}
    \label{fig:ROC-1}
\end{figure}

\begin{figure}[H]
    \centering
    \includegraphics[width=\linewidth]{r_test_dokładności/AUC_charts/2.png}
    \caption{Wykres krzywej ROC i wartość AUC dla poziomu oświetlenia nr 2.}
    \label{fig:ROC-2}
\end{figure}

\begin{figure}[H]
    \centering
    \includegraphics[width=\linewidth]{r_test_dokładności/AUC_charts/3.png}
    \caption{Wykres krzywej ROC i wartość AUC dla poziomu oświetlenia nr 3.}
    \label{fig:ROC-3}
\end{figure}

\begin{figure}[H]
    \centering
    \includegraphics[width=\linewidth]{r_test_dokładności/AUC_charts/4.png}
    \caption{Wykres krzywej ROC i wartość AUC dla poziomu oświetlenia nr 4.}
    \label{fig:ROC-4}
\end{figure}

\begin{figure}[H]
    \centering
    \includegraphics[width=\linewidth]{r_test_dokładności/AUC_charts/porownanieAUC.png}
    \caption{Wykres porównujący wartości AUC dla wszystkich poziomów oświetlenia.}
    \label{fig:AUC}
\end{figure}

Przed wyciągnięciem wniosków z wykresów należy zinterpretować jakościowo wszystkie poziomy oświetlenia. 
\begin{itemize}
    \item \textbf{Poziom nr 1:} Obiekty są bardzo dobrze widoczne. Limonkowe tło pokoju korzystnie wpływa na podkreślenie konturów człowieka. Sprzyja także nasycenie barwy. Oczekuje się i wymaga dobrego wyniku detektora. 
    \item \textbf{Poziom nr 2:} Widoczność równie dobra co dla poziomu nr 1. Różnica nasycenia oraz jasności pomiędzy tymi poziomami jest zdecydowanie mniejsza niż np. róznica miedzy poziomami nr 2 i 3. Oczekuje się i wymaga podobnego wyniku co w przypadku poziomu nr 1.
    \item \textbf{Poziom nr 3:} Poziom ten stanowi wyraźny przeskok w widoczności w porównaniu do poprzednich poziomów. Mimo to ludzkie oko dostrzega obecność człowieka w każdej klatce, kiedy jest on uznany za obecnego. Poziom ten uznaje się za najważniejszy w badaniu.
    \item \textbf{Poziom nr 4:} Obiekty są bardzo słabo widoczne. Dla wielu zdefiniowanych klatek samo ludzkie oko ma problem ze stwierdzeniem obecności człowieka w pomieszczeniu. Z racji tego, przewiduje się złe osiągi modelu. Jest to mało prawdopodobne, iż model dokładniej stwierdzi obecność obiektu niż ludzkie oko. W związku z tym poziom ten nie jest znaczący w analizie i finalnej ocenie modelu dla tego badania. 
\end{itemize}

Jak widać na wykresach na rysunkach \ref{fig:ROC-1} i \ref{fig:ROC-2} model osiągnął wartość idealną, równą 1. Budzi to zastrzeżenia co do poprawności przygotowanych danych. W rzeczywistośći wartość AUC nie jest równa 1, lecz bardzo bliska 1, więc wartość ta jest efektem zaokrąglenia. Można więc stwierdzić, że w bardzo dobrych warunkach oświetlenia model sprawdza się wzorowo. Jest to przydatna informacja, jeśli użytkownik uważa, że może wykorzystać system do celów, w których istotna jest bliska zeru liczba błędnych reakcji systemu. ZZ uwagi na prosty scenariusz testowy oraz brak dalszych badań, nie rekomenduje się jednak używania systemu w scenariuszach krytycznych, takich jak monitoring wizyjny.

Teoretyczną wartość dla poziomu nr 3 można uznać za dobrą. W praktyce można interpretować ją różnie. Skalarna wartość jest jednak bardzo ogólną metryką, a jej interpretacja w dużej mierze zależy od konkretnego ustawienia modelu. 

Wyniki poziomu nr 4 okazały się, zgodnie z przewidywaniami, bardzo słabe. Pokazuje to, że w takich warunkach oświetleniowych system ten nie powinien być używany.

Warto podkreślić, iż natura krzywej ROC jest zawsze niemalejąca. Oznacza to, iż zwiększenie czułości na detekcję obiektu jest sprzężone z generacją wiekszej liczby fałszywych alarmów.
Pokazuje to też skomplikowianie problemu doboru odpowiedniego progu ufności do zadania -- manipulacja progiem jest związana ze zmianą wartości jednej metryki, a ta z kolei wpływa na drugą.

Podsumowując, przeprowadzone porównanie potwierdziło spodziewane zachowanie -- coraz gorsze wyniki w miarę pogarszających się warunków oświetleniowych, co stanowi dowód na stabilność modelu. 
Co więcej, na podstawie krzywej ROC stwierdzono złożoność problemu doboru odpowiedniego progu ufności. 
Stwierdzono również, że dalsze badania dla poziomu oświetlenia nr 4 są zbędne. 
Zademonstrowano też, iż system bardzo dobrze sprawdza się w dobrych warunkach oświetleniowych. 
Taki system mógłby być używany np. w sytuacji, gdy obiektyw kamery jest skierowany na wejście do sklepu, powiadamiając pracowników na zapleczu o konieczności obsługi klienta. 
TJak jednak podkreślono, AUC jest miarą bardzo ogólną, dlatego w celu dokładniejszego stwierdzenia użyteczności modelu, należy zbadać jego wyniki w zależności od ustawień systemu (np. progu ufności -- następne badanie). Samo stosowanie krzywej ROC oraz wartości AUC w ewaluacji modeli budzi kontrowersje i wątpliwości, które zostały omówione w pracach \cite{AUC_critique1,AUC_critique2}.






\subsection{Analiza wpływu progu ufności}
W tej sekcji skupiono się na zbadaniu optymalnego ustawienia progu ufności w aplikacjach systemu, gdzie powinien on pełnić role asystenta. W tym kontekście priorytetem jest największa redukcja wyników fałszywych pozytywnych (FP). 

Opierając się na wnioskach z poprzedniego badania, do dalszych testów nie wykorzystano filmu z poziomem oświetlenia nr 4. 
Ponadto, z racji podobnych wyników dla poziomów nr 1 i nr 2 zdecydowano, iż test ten użyje tylko poziomu nr 1 (przykład dobrego oświetlenia). Przeanalizowano również wyniki dla poziomu nr 3 jako przykład problematycznego oświetlenia. 

Poziom 1 odpowiadać może przytoczonemu powyżej użyciu w sklepie --- minimalizacja sytuacji kiedy pracownik wychodzi z zaplecza do nieobecnego w rzeczywistości klienta.
Natomiast poziom nr 3 może odpowiadać sytuacji lekko oświetlonego wejścia do domu w nocy --- minimalizacja sytuacji, kiedy użytkownik jest niepotrzebnie budzony przez alarm. Zestawy danych (klatek) nie są zbalansowane ani względem filmów, ani względem klatek z obecnym i nieobecnym obiektem. Dlatego też nie zostały użyte dalsze metryki i skupiono się analizie podstawowych metryk klasyfikacji binarnej (TP, TN, FP, FN). 
Tabela \ref{tab:liczba-klatek} obrazuje liczbę klatek z obecnym oraz nieobecnym człowiekiem.
\begin{table}[H]
    \centering
    \caption{Liczba klatek w filmach z wyróżnieniem klatek zawierających i nie zawierających badaną klase.}
    \begin{tabular}{c|ccc|}
    \cline{2-4}
                                             & \multicolumn{3}{c|}{Liczba klatek}                                                         \\ \hline
    \multicolumn{1}{|c|}{Poziom oświetlenia} & \multicolumn{1}{c|}{Człowiek obecny} & \multicolumn{1}{c|}{Człowiek nieobecny} & Całkowita \\ \hline
    \multicolumn{1}{|c|}{1}                  & \multicolumn{1}{c|}{100}             & \multicolumn{1}{c|}{122}                & 222       \\ \hline
    \multicolumn{1}{|c|}{3}                  & \multicolumn{1}{c|}{61}              & \multicolumn{1}{c|}{41}                 & 102       \\ \hline
    \end{tabular}
    \label{tab:liczba-klatek}
    \end{table}

Mówiąc o wpływie progu ufności, wraz z jego wzrostem należy spodziewać się mniejszej liczby FP i większej liczby TN w przypadku obecności obiektu oraz mniejszej liczby TP i większej liczby FN  w przypadku jego nieobecności. Zależność tę obrazują wykresy na rysunkach \ref{fig:binary-1} i \ref{fig:binary-3}.
\begin{figure}[H]
    \centering
    \includegraphics[width=\linewidth]{r_test_dokładności/binary_charts/1.png}
    \caption{Wykres zestawiający liczbę wystąpień metryk klasyfikacji binarnej dla poziomu oświetlenia nr 1.}
    \label{fig:binary-1}
\end{figure}
\begin{figure}[H]
    \centering
    \includegraphics[width=\linewidth]{r_test_dokładności/binary_charts/3.png}
    \caption{Wykres zestawiający liczbę wystąpień metryk klasyfikacji binarnej dla poziomu oświetlenia nr 3.}
    \label{fig:binary-3}
\end{figure}
Na podstawie powyższych wykresów można dostosować próg ufności w taki sposób, aby system jak najlepiej odpowiadał założonemu scenariuszowi. Bazując na przedstawionej zależności, należy znaleźć najmniejszą wartość progu ufności, dla której FP jest wystarczająco małe, a TP na odpowiednio wysokim poziomie. Ustalenie progu, przy którym FP jest bliskie zeru, nie zawsze prowadzi do dobrych rezultatów, ponieważ może to skutkować zbyt niską czułością systemu, co czyni go mało użytecznym.

Zaczynając od analizy poziomu nr 1, próg taki okazał się być bardzo niski -- równy $0.04$. Jest to sytuacja idalna, gdyż udało się uzyskać FP na poziomie 0 ($0\%$) oraz TP na poziomie 100 ($100\%$). Potwierdza to wnioski z badania w podrozdziale \ref{sec:test-AUC} -- system ten jest bardzo dobrym wyborem do wykorzystania w bardzo dobrych warunkach oświetleniowych.

Przechodząc do poziomu nr 3, w przypadku chęci utrzymania fałszywych alarmów na poziomie $0$ (próg ufności równy $0.09$) otrzymany wynik TP to $32$ ($52,46\%$). Jednoznaczne stwierdzenie czy wynik TP jest wystarczający jest niemożliwe --  konieczna byłaby dalsza analiza okoliczności. Na nagranym filmie człowiek przybliża się do kamery na pewną odległość (do tego momentu zawsze twarzą do obiektywu) po czym wykonuje obrót wokół własnej osi.
Model mógł np. wykryć tylko sytuacje gdy człowiek był skierowany twarzą do obiektywu. Zakładając, że człowiek zawsze będzie skierowany twarzą do obiektywu oznacza, że wynik taki jest satysfakcjonujący, ponieważ model poradził sobie z różnym rozmiarem klasy obiektu na obrazie. Natomiast w przykładowej sytuacji kiedy model nie potrafił wykryć gdy człowiek znajdował się dalej od obiektywu, hipotetyczny mieszkaniec domu nie zostałby poinformowany o napotkanym intruzie, co określa wynik TP jako niesatysfakcjonujący. Bez dalszych badań, po pierwsze, na bardziej zbalansowanym zestawie danych, po drugie, dla badania bardziej konkretnego typu sceny np. z rozróżnieniem na odległość obiektu od kamery nie da się stwierdzić optymalnego progu ufności.

Podsumowując, przeprowadzone badania udowodniły skuteczność YOLOv8n w warunkach bardzo dobrego oświetlenia dla prostych scenariuszów testowych z pojedynczą liczbą obiektów na ekranie. Interesującym okazała się bardzo niska wartość progu ufności, dla którego stwierdzono optymalne ustawienie. Kwestia ta oraz niejednoznaczne wyniki dla innych poziomów oświetlenia rodzi wiele pytań oraz jest potencjalnym punktem dalszych badań w przyszłości.







\section{Test 2}
W tym podrozdziale zbadano jak dobrze model radzi sobie z różnymi obiektami należącymi do tej samej klasy. Umiejętność generalizacji zostanie sprawdzona na przykładzie klasy \emph{krzesło} (ang. \emph{chair}). Użyte obiekty to krzesło oraz fotel dla graczy (dalej zwany fotelem). Nazwy angielskie tych obiektów to kolejno \emph{chair} i \emph{gaming chair}, co jest istotne, ponieważ klasy COCO zostały stworzone według angielskiego nazewnictwa, na podstawie którego przypisuje się różne warianty obiektów do jednej klasy. W badaniu tym porównano wyniki dla krzesła i fotelu w funkcji progu ufności dla wszystkich czterech poziomów oświetlenia. Do testu wykorzystano osiem filmów -- cztery z widocznym krzesłem i człowiekiem, oraz cztery z widocznym fotelem i człowiekiem. Wartości jasności i nasycenia dla każdego filmu ukazono w tabeli \ref{tab:jasnosc-krzeslo-fotel}. Wygląd nagranego pomieszczenia wraz z nagranym obiektem dla różnych poziomów oświetlenia ukazano na rysunku \ref{fig:chair_grid} (człowiek, krzesło) i 
\ref{fig:game_grid} (człowiek fotel). 
% Please add the following required packages to your document preamble:
% \usepackage{multirow}
\begin{table}[H]
    \centering
    \caption{Jasność i nasycenie dla wszystkich filmów. Pogrubiona czcionka oznacza obiekt należący do analizowanej klasy.}
    \begin{tabular}{|c|c|c|c|}
    \hline
    Poziom   oświetlenia & Obecne obiekty                                & Jasność & Nasycenie \\ \hline
    \multirow{2}{*}{1}   & człowiek,   
    \textbf{krzesło} & 149.92  & 119.07    \\ \cline{2-4} 
                         & człowiek, \textbf{fotel}     & 152.68  & 111.47    \\ \hline
    \multirow{2}{*}{2}   & człowiek,   \textbf{krzesło} & 139.14  & 90.71     \\ \cline{2-4} 
                         & człowiek, \textbf{fotel}     & 133.77  & 83.29     \\ \hline
    \multirow{2}{*}{3}   & człowiek,   \textbf{krzesło} & 38.91   & 132.31    \\ \cline{2-4} 
                         & człowiek, \textbf{fotel}     & 38.12   & 124.31    \\ \hline
    \multirow{2}{*}{4}   & człowiek,   \textbf{krzesło} & 25.3    & 100.8     \\ \cline{2-4} 
                         & człowiek, \textbf{fotel}     & 24.88   & 108.5     \\ \hline
    \end{tabular}
    \label{tab:jasnosc-krzeslo-fotel}
    \end{table}

Tak jak wspomniano (podrozdział \ref{sec:zrodlo_wideo}), badane obiekty są statyczne i były obecne przez wszystkie klatki każdego filmu. Dlatego też metryki FP i TN są zawsze zerowe. Do tego badania można jednak ponownie skorzystać z TPR, ponieważ metryka ta bazuje jedynie na TP i FN. TPR jest alternatywnie nazywana czułością i to badanie tak będzie się do niej odwoływać. Czułość dla fotela i krzesła jest zestawiona w funkcji progu ufności dla kolejnych poziomów oświetlenia (wykresy na rysunkach \ref{fig:chair-game-1}, \ref{fig:chair-game-2}, \ref{fig:chair-game-3}, \ref{fig:chair-game-4}). Porównano również poziomy oświetlenia dla konkretnego obiektu: \ref{fig:all_bright_chair} (krzesło), \ref{fig:all_bright_game} (fotel). 

Wyniki dla każdego poziomu oświetlenia okazały się być korzystniejsze dla krzesła. Przyczyn można upatrywać w dwóch źródłach. Po pierwsze, zdecydowanie ciemniejszy kolor fotela jest ciężej wykrywany w coraz ciemniejszym pomieszczeniu. Po drugie, model COCO mógł być wytrenowany dla mniejszej liczby foteli tego typu.

Odnośnie przyczyny pierwszej, można dostrzec zauważalny spadek skuteczeności klasyfikacji na niższych poziomach oświetlenia od poziomu 1. Stało się tak, pomimo że poziom 2 został uznany za poziom dobrej widoczności -- fotel jest wyraźnie wyszczególniony. Przyczyną może być np. niska jakość nagranego filmu, w tym zakłócenia. 

Podsumowując, stwierdzono, iż model nie generalizuje dobrze klasy \emph{krzesło}. Samo badanie nie dowodzi, że generalizacja YOLOv8n jest podobnej jakości dla innych klas. Interesującym rozszerzeniem badań byłoby zbadanie klas niestatycznych, czego jednak nie udało się zrealizować z powodu osobistego braku takich obiektów.

\begin{figure}[H]
    \centering
    \includegraphics[width=\linewidth]{r_test_dokładności/chair_charts/1.png}
    \caption{Wykres zestawiający czułość w funkcji progu ufności dla krzesła i fotela. Poziom oświetlania nr 1.}
    \label{fig:chair-game-1}
\end{figure}

\begin{figure}[H]
    \centering
    \includegraphics[width=\linewidth]{r_test_dokładności/chair_charts/2.png}
    \caption{Wykres zestawiający czułość w funkcji progu ufności dla krzesła i fotela. Poziom oświetlania nr 2.}
    \label{fig:chair-game-2}
\end{figure}

\begin{figure}[H]
    \centering
    \includegraphics[width=\linewidth]{r_test_dokładności/chair_charts/3.png}
    \caption{Wykres zestawiający czułość w funkcji progu ufności dla krzesła i fotela. Poziom oświetlania nr 3.}
    \label{fig:chair-game-3}
\end{figure}

\begin{figure}[H]
    \centering
    \includegraphics[width=\linewidth]{r_test_dokładności/chair_charts/4.png}
    \caption{Wykres zestawiający czułość w funkcji progu ufności dla krzesła i fotela. Poziom oświetlania nr 4.}
    \label{fig:chair-game-4}
\end{figure}

\begin{figure}[H]
    \centering
    \includegraphics[width=\linewidth]{r_test_dokładności/chair_charts/chair.png}
    \caption{Wykres zestawiający czułość w funkcji progu ufności dla wszystkich poziomów oświetlenia. Obiekt -- krzesło.}
    \label{fig:all_bright_chair}
\end{figure}

\begin{figure}[H]
    \centering
    \includegraphics[width=\linewidth]{r_test_dokładności/chair_charts/gaming-chair.png}
    \caption{Wykres zestawiający czułość w funkcji progu ufności dla wszystkich poziomów oświetlenia. Obiekt --  fotel.}
    \label{fig:all_bright_game}
\end{figure}

