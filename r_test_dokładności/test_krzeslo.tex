W tym podrozdziale zbadano jak dobrze model radzi sobie z różnymi obiektami należącymi do tej samej klasy. Umiejętność generalizacji zostanie sprawdzona na przykładzie klasy \emph{krzesło} (ang. \emph{chair}). Użyte obiekty to krzesło oraz fotel dla graczy (dalej zwany fotelem). Nazwy angielskie tych obiektów to kolejno \emph{chair} i \emph{gaming chair}, co jest istotne, ponieważ klasy COCO zostały stworzone według angielskiego nazewnictwa, na podstawie którego przypisuje się różne warianty obiektów do jednej klasy. W badaniu tym porównano wyniki dla krzesła i fotelu w funkcji progu ufności dla wszystkich czterech poziomów oświetlenia. Do testu wykorzystano osiem filmów -- cztery z widocznym krzesłem i człowiekiem, oraz cztery z widocznym fotelem i człowiekiem. Wartości jasności i nasycenia dla każdego filmu ukazono w tabeli \ref{tab:jasnosc-krzeslo-fotel}. Wygląd nagranego pomieszczenia wraz z nagranym obiektem dla różnych poziomów oświetlenia ukazano na rysunku \ref{fig:chair_grid} (człowiek, krzesło) i 
\ref{fig:game_grid} (człowiek fotel). 
% Please add the following required packages to your document preamble:
% \usepackage{multirow}
\begin{table}[H]
    \centering
    \caption{Jasność i nasycenie dla wszystkich filmów. Pogrubiona czcionka oznacza obiekt należący do analizowanej klasy.}
    \begin{tabular}{|c|c|c|c|}
    \hline
    Poziom   oświetlenia & Obecne obiekty                                & Jasność & Nasycenie \\ \hline
    \multirow{2}{*}{1}   & człowiek,   
    \textbf{krzesło} & 149.92  & 119.07    \\ \cline{2-4} 
                         & człowiek, \textbf{fotel}     & 152.68  & 111.47    \\ \hline
    \multirow{2}{*}{2}   & człowiek,   \textbf{krzesło} & 139.14  & 90.71     \\ \cline{2-4} 
                         & człowiek, \textbf{fotel}     & 133.77  & 83.29     \\ \hline
    \multirow{2}{*}{3}   & człowiek,   \textbf{krzesło} & 38.91   & 132.31    \\ \cline{2-4} 
                         & człowiek, \textbf{fotel}     & 38.12   & 124.31    \\ \hline
    \multirow{2}{*}{4}   & człowiek,   \textbf{krzesło} & 25.3    & 100.8     \\ \cline{2-4} 
                         & człowiek, \textbf{fotel}     & 24.88   & 108.5     \\ \hline
    \end{tabular}
    \label{tab:jasnosc-krzeslo-fotel}
    \end{table}

Tak jak wspomniano (podrozdział \ref{sec:zrodlo_wideo}), badane obiekty są statyczne i były obecne przez wszystkie klatki każdego filmu. Dlatego też metryki FP i TN są zawsze zerowe. Do tego badania można jednak ponownie skorzystać z TPR, ponieważ metryka ta bazuje jedynie na TP i FN. TPR jest alternatywnie nazywana czułością i to badanie tak będzie się do niej odwoływać. Czułość dla fotela i krzesła jest zestawiona w funkcji progu ufności dla kolejnych poziomów oświetlenia (wykresy na rysunkach \ref{fig:chair-game-1}, \ref{fig:chair-game-2}, \ref{fig:chair-game-3}, \ref{fig:chair-game-4}). Porównano również poziomy oświetlenia dla konkretnego obiektu: \ref{fig:all_bright_chair} (krzesło), \ref{fig:all_bright_game} (fotel). 

Wyniki dla każdego poziomu oświetlenia okazały się być korzystniejsze dla krzesła. Przyczyn można upatrywać w dwóch źródłach. Po pierwsze, zdecydowanie ciemniejszy kolor fotela jest ciężej wykrywany w coraz ciemniejszym pomieszczeniu. Po drugie, model COCO mógł być wytrenowany dla mniejszej liczby foteli tego typu.

Odnośnie przyczyny pierwszej, można dostrzec zauważalny spadek skuteczeności klasyfikacji na niższych poziomach oświetlenia od poziomu 1. Stało się tak, pomimo że poziom 2 został uznany za poziom dobrej widoczności -- fotel jest wyraźnie wyszczególniony. Przyczyną może być np. niska jakość nagranego filmu, w tym zakłócenia. 

Podsumowując, stwierdzono, iż model nie generalizuje dobrze klasy \emph{krzesło}. Samo badanie nie dowodzi, że generalizacja YOLOv8n jest podobnej jakości dla innych klas. Interesującym rozszerzeniem badań byłoby zbadanie klas niestatycznych, czego jednak nie udało się zrealizować z powodu osobistego braku takich obiektów.

\begin{figure}[H]
    \centering
    \includegraphics[width=\linewidth]{r_test_dokładności/chair_charts/1.png}
    \caption{Wykres zestawiający czułość w funkcji progu ufności dla krzesła i fotela. Poziom oświetlania nr 1.}
    \label{fig:chair-game-1}
\end{figure}

\begin{figure}[H]
    \centering
    \includegraphics[width=\linewidth]{r_test_dokładności/chair_charts/2.png}
    \caption{Wykres zestawiający czułość w funkcji progu ufności dla krzesła i fotela. Poziom oświetlania nr 2.}
    \label{fig:chair-game-2}
\end{figure}

\begin{figure}[H]
    \centering
    \includegraphics[width=\linewidth]{r_test_dokładności/chair_charts/3.png}
    \caption{Wykres zestawiający czułość w funkcji progu ufności dla krzesła i fotela. Poziom oświetlania nr 3.}
    \label{fig:chair-game-3}
\end{figure}

\begin{figure}[H]
    \centering
    \includegraphics[width=\linewidth]{r_test_dokładności/chair_charts/4.png}
    \caption{Wykres zestawiający czułość w funkcji progu ufności dla krzesła i fotela. Poziom oświetlania nr 4.}
    \label{fig:chair-game-4}
\end{figure}

\begin{figure}[H]
    \centering
    \includegraphics[width=\linewidth]{r_test_dokładności/chair_charts/chair.png}
    \caption{Wykres zestawiający czułość w funkcji progu ufności dla wszystkich poziomów oświetlenia. Obiekt -- krzesło.}
    \label{fig:all_bright_chair}
\end{figure}

\begin{figure}[H]
    \centering
    \includegraphics[width=\linewidth]{r_test_dokładności/chair_charts/gaming-chair.png}
    \caption{Wykres zestawiający czułość w funkcji progu ufności dla wszystkich poziomów oświetlenia. Obiekt --  fotel.}
    \label{fig:all_bright_game}
\end{figure}