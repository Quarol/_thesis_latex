\chapter{Wprowadzenie}
Celem niniejszej pracy jest stworzenie systemu do detekcji obiektów w czasie rzeczywistym na obrazach z kamery. Zadaniem systemu będzie audio-wizualne alarmowanie użytkownika w momencie pojawienia się wykrywanego obiektu w obiektywie kamery. System jest oprogramowaniem dostępnym na komputer, wykorzystującym podłączoną do niego kamerę. Realizację zadania w czasie rzeczywistym rozumie się poprzez użycie kamery jako źródła strumienia klatek obrazu. 

Oprogramowanie realizuje funkcje odpowiedzialne za pobór klatek ze źródła wideo, detekcję obiektów oraz, w sytuacji wykrycia obiektów, przetworzenie pobranej klatki w celu wizualizacji tych obiektów. Ponadto, realizowana jest funkcja uruchomienia pliku dźwiękowego, po wykryciu obiektów przez detektor, oraz funkcja wyświetlania kolejnych klatek obrazu, w tym klatek bez wykrytych obiektów (klatek nieprzetworzonych) i z wykrytymi obiektami (klatek przetworzonych). Zadanie alarmowania jest w tym kontekście rozumiane jako uruchomienie alarmu dźwiękowego oraz wyświetlenie przetworzonej klatki obrazu. Oprogramowanie zawiera graficzny interfejs użytkownika w celu wyświetlenia klatek obrazu oraz umożliwienia użytkownikowi zmiany parametrów detektora.

System wykorzystuje istniejące rozwiązania do implementacji w.w. celów. Zakresem pracy jest implementacja oprogramowania zgodnie w opisanymi wymaganiami oraz sprawdzenie szybkości systemu dla różnych ustawień kamery. W zakres wchodzi również przeprowadzenie badań skuteczności wybranego modelu detekcji obiektów w autorskim scenariuszu testowym.