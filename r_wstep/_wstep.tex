\chapter{Wprowadzenie}


Celem niniejszej pracy jest stworzenie systemu do detekcji obiektów w czasie rzeczywistym na obrazach z kamery. Zadaniem systemu będzie audio-wizualne alarmowanie użytkownika w momencie pojawienia się wykrywanego obiektu w obiektywie kamery. System będzie oprogramowaniem dostępnym na komputer, wykorzystującym podłączoną do niego kamerę. Realizację zadania w czasie rzeczywistym rozumie się poprzez użycie kamery jako źródła strumienia klatek obrazu. 

Oprogramowanie będzie realizować funkcje odpowiedzialne za pobór klatek ze źródła wideo, detekcję obiektów oraz, w sytuacji wykrycia obiektów, przetworzenie pobranej klatki w celu wizualizacji tych obiektów. Ponadto, będzie realizowana funkcja uruchomienia pliku dźwiękowego, po wykryciu obiektów przez detektor, oraz funkcja wyświetlania kolejnych klatek obrazu, w tym klatek bez wykrytych obiektów (klatek nieprzetworzonych) i z wykrytymi obiektami (klatek przetworzonych). Zadanie alarmowania jest w tym kontekście rozumiane jako uruchomienie alarmu dźwiękowego oraz wyświetlenie przetworzonej klatki obrazu. Oprogramowanie będzie zawierać graficzny interfejs użytkownika w celu wyświetlenia klatek obrazu oraz umożliwienia użytkownikowi zmiany wybranych parametrów detektora.

System będzie wykorzystywać istniejące rozwiązania do wykonania w.w. celów. Zakresem pracy jest implementacja oprogramowania zgodnie w opisanymi wymaganiami oraz sprawdzenie szybkości systemu dla różnych ustawień rodzielczości kamery. 
W zakres wchodzi również przeprowadzenie badań skuteczności wybranego modelu detekcji obiektów w zależności od różnych poziomów oświetlenia w lokalizacji umiejscowienia kamery oraz manipulacji wartością parametru progu ufności modelu. Zbiorem danych badawczych będą autorskie nagrania z przykładowego scenariusza użycia systemu. Model zostanie zbadany pod kątem skuteczności klasyfikacji obiektów na przykładzie wybranych klas. Przetestowana zostanie m.in. zdolność generalizacji modelu.     