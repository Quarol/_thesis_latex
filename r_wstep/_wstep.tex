\chapter{Wprowadzenie}
Dynamiczny rozwój wizji komputerowej przyczynił się do upowszechnienia technologii z tej dziedziny wśród przeciętnych użytkowników. Detekcja obiektów była niegdyś kojarzona z dużymi systemami, które wykorzystywały skomplikowane oraz drogie komponenty. Aplikacje krytyczne, takie jak pojazdy autonomiczne, nadal wymagają takich rozwiązań, aby zminimalizować ryzyko kosztownych w skutkach błędów. Niemniej w wielu przypadkach udało się zredukować potrzebę zastosowania specjalistycznego sprzętu komputerowego. Dzięki rozwiązaniom takim jak modele oparte na konwolucyjnych sieciach neuronowych z rodziny YOLO możliwe stało się wydajne i skuteczne wykrywanie obiektów na urządzeniach posiadanych przez przeciętnego użytkownika jak np. laptopy oraz telefony komórkowe. Rozwój dziedziny oraz coraz większa dostępność kamer, często wbudowanych w posiadane urządzenia, pozwala na użytkowanie systemów działających w czasie rzeczywistym bez konieczności zakupu drogich, specjalistycznych rozwiązań. 

Niniejszy projekt zakłada stworzenie oprogramowania do detekcji obiektów w czasie rzeczywistym, przeznaczonego do ogólnych zastosowań, z wykorzystaniem popularnych i powszechnie dostępnych, niespecjalistycznych zasobów sprzętowych -- kamery, procesora oraz procesora graficznego. Zakłada się również przeprowadzenie badań nad wytworzonym oprogramowaniem.

Celem systemu będzie audio-wizualne alarmowanie użytkownika w momencie wykrycia obiektu w obiektywie kamery w czasie rzeczywistym. Zakłada się, iż system będzie pełnić rolę asystenta w zadaniu wykrycia obiektów. 
System będzie oprogramowaniem dostępnym na komputer, wykorzystującym podłączoną do niego kamerę. Realizację zadania w czasie rzeczywistym rozumie się poprzez użycie kamery jako źródła strumienia klatek obrazu. 

Oprogramowanie ma za zadanie synchronizować działanie kamery z modułem detekcji obiektów. Aplikacja będzie zawierać graficzny interfejs użytkownika w celu wyświetlenia klatek obrazu z kamery oraz umożliwienia użytkownikowi zmiany wybranych parametrów detekcji obiektów --- wykrywane klasy oraz próg ufności detektora. Alarmowanie zdefiniowane jest poprzez wizualizację lokalizacji obiektów na wyświetlanych klatkach oraz uruchomienie alarmu dźwiękowego dla klatek obrazu z wykrytymi obiektami. System wykorzystuje model YOLOv8n do przeprowadzenia detekcji obiektów. Model umożliwia manipulację w.w. parametrami detekcji oraz detekcję obiektów na dostępnym zbiorze osiemdziesięciu klas.  

Zakresem pracy jest implementacja oprogramowania zgodnie z opisanymi wymaganiami oraz sprawdzenie szybkości wykonanego systemu -- rozumianego jako czas potrzebny do wyświetlenia klatki obrazu -- dla dostępnych ustawień rozdzielczości użytej kamery.   
W zakres wchodzi również przeprowadzenie badań skuteczności wykorzystanego modelu detekcji obiektów w zależności od różnych poziomów oświetlenia w lokalizacji umiejscowienia kamery oraz manipulacji wartością parametru progu ufności modelu. Zbiorem danych badawczych będą autorskie nagrania z przykładowego scenariusza użycia systemu. Model zostanie zbadany pod kątem skuteczności klasyfikacji obiektów na przykładzie wybranych klas. Testy obejmą m.in. zdolność modelu do generalizacji obiektów oraz wpływ progu ufności na wyniki detekcji.