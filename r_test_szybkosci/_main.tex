\chapter{Test szybkości systemu}
\label{chap:test-szybkosci}
W tym rozdziale opisano badanie sprawdzające ile czasu zajmuje systemowi wykonanie wszystkich zadań koniecznych do wyświetlenia klatki obrazu. Porównano w nim m.in. wpływ rozdzielczości kamery. Pomiary zrealizowano dla trzech różnych ustawień rozdzielczości tej samej kamery: 640x360px, 640x480px i 1280x720px.
Według informacji pobranych z kamery za pomocą biblioteki OpenCV, operuje ona w 30 klatkach na sekundę, co daje około 33.33 milisekund na pojedyńczą klatkę. Liczba klatek użytych do testu to pięć tysięcy dla każdej rodzielczości.
Test wykonano na komputerze wyposażonym w kartę graficzną \emph{NVIDIA GeForce GTX 1650} i procesor \emph{Intel Core i5-8300H 2.30GHz}.

Poprzez czas potrzebny do wyświetlenia klatki rozumie się szereg czynności. Jest to proces rozpoczynający się od pobrania klatki z kamery, następnie detekcji obiektów oraz przetworzenia klatki w celu oznaczenia wykrytych obiektów, a kończący się poprzez wyświetlenie przetworzonej klatki obrazu przez graficzny interfejs użytkownika. 

W teście zmierzono całkowity czas do wyświetlenia wszystkich klatek oraz czas potrzebny do wyświetlenia pojedyńczej klatki (pięć tysięcy pomiarów na każdą rozdzielczość). Czas zmierzono poprzez pobranie w dwóch punktach -- początkowym i końcowym -- kodu źródłowego aktualnego czasu w systemie operacyjnym i obliczeniu różnicy jaka wystąpiła między tymi punktami. 

Dla pomiaru czasu całkowitego punkt początowy umieszczono po wyświetleniu przez GUI pierwszej klatki (klatka zerowa --- niewliczona do pomiaru). Punkt końcowy zmierzono po wyświetleniu ostatniej klatki.
Dla pojedyńczego pomiaru punkt początkowy jest taki sam dla pierwszej uwzględnionej klatki. Punkt końcowy jest ustawiany po wyświetleniu następnej klatki, w tym momencie też punkt końcowy staje się punktem początkowym dla pomiaru dla kolejnej klatki.
Pomiary czasu całkowitego dla zbadanych rozdzielczości kamery przedstawiono w tabeli \ref{tab:czas-calkowity-5000klatek}. Tabela uwzględnia zmierzony czas całkowity oraz obliczony na jej podstawie średni czas na klatkę oraz liczbę klatek na sekundę (FPS). Średni czas na klatkę wyznaczono poprzez podzielenie czasu całkowitego przez liczbę klatek (5000), natomiast FPS obliczono, dzieląc liczbę klatek przez czas całkowity.  
\begin{table}[H]
\centering
\caption{Pomiar czasu potrzebnego do wyświetlenia 5000 klatek.}
\begin{tabular}{|c|c|c|c|}
\hline
Rozdzielczość & Średni czas na klatke {[}ms{]} & Czas całkowity {[}s{]} & FPS   \\ \hline
640x360px     & 33.34                          & 166.69                 & 30    \\ \hline
640x480px     & 33.48                          & 167.42                 & 29.87 \\ \hline
1280x720px    & 33.58                          & 167.89                 & 29.78 \\ \hline
\end{tabular}
\label{tab:czas-calkowity-5000klatek}
\end{table}

Pomiary czasu całkowitego wykazały brak wpływu rozdzielczości kamery na szybkość systemu. Przyczynę takiego zachowania można upatrywać w optymalizacji YOLOv8, ponieważ model ten przed rozpoczęciem detekcji skaluje obraz wejściowy do stałego rozmiaru -- do domyślnego rozmiaru lub ustawionego przez programistę za pomocą parametru inferencji \emph{imgsz} -- i wykonuje detekcję na przeskalowanym obrazie. Z racji braku opisu w dokumentacji modelu \cite{yolo_docs}, potwierdzenie istnienia tego mechanizmu opisano jako odpowiedź do pytania zadanego na forum, które jest umieszczone razem z repozytorium kodu modelu \cite{github_imgsz}.

Interesującą obserwacją jest również osiągnięcie liczby klatek na sekundę równej FPS kamery. Powodem może być np. wariacja w liczbie rzeczywistych klatek na sekundę dostarcznych przez kamerę lub OpenCV (np. powtórzenie odczytu tej samej klatki).

Pomiary czasu całkowitego nie obrazują jednak potencjalnych odchyleń od średniej. Dlatego postanowiono zilustrować pojedyncze pomiary. Wykresy punktowe na rysunkach \ref{fig:czas-punktowy1}, \ref{fig:czas-punktowy2} i \ref{fig:czas-punktowy3} przedstawiają te pomiary  i zestawiają je z odpowiadającym średnim czasem z tabeli \ref{tab:liczba-klatek}. Aby sprawdzić różnicę w odchyleniach pomiędzy rozdzielczościami, dane z powyżej wspomnianych wykresów scalono na wykresie na ryskunku \ref{fig:czas-punktowy-all}.  


\begin{figure}[H]
    \centering
    \includegraphics[width=\linewidth]{r_test_szybkosci/punkty/1.png}
    \caption{Wykres punktowy dla rodzielczości 640x360px przedstawiający pomiary czasu do wyświetlenia obrazu dla każdej klatki.}
    \label{fig:czas-punktowy1}    
\end{figure}

\begin{figure}[H]
    \centering
    \includegraphics[width=\linewidth]{r_test_szybkosci/punkty/2.png}
    \caption{Wykres punktowy dla rodzielczości 640x480px przedstawiający pomiary czasu do wyświetlenia obrazu dla każdej klatki.}
    \label{fig:czas-punktowy2}    
\end{figure}

\begin{figure}[H]
    \centering
    \includegraphics[width=\linewidth]{r_test_szybkosci/punkty/3.png}
    \caption{Wykres punktowy dla rodzielczości 1280x720px przedstawiający pomiary czasu do wyświetlenia obrazu dla każdej klatki.}
    \label{fig:czas-punktowy3}    
\end{figure}

\begin{figure}[H]
    \centering
    \includegraphics[width=\linewidth]{r_test_szybkosci/punkty/all.png}
    \caption{Wykres punktowy dla wszystich rozdzielczości przedstawiający pomiary czasu do wyświetlenia obrazu dla każdej klatki.}
    \label{fig:czas-punktowy-all}    
\end{figure}

Dla wyników dla każdej rozdzielczości można zaobserwować zagęsczenie punktów przy średniej wartości. 
Stopień zagęszczenia pomiędzy rozdzielczościami można ocenić na podobny, chociaż widoczna jest większa intenstywność zagęszczenia dla rozdzielczośc 640x480px. Może być to spowodowane tym, że jest to ustawienie najbliżej odpowiadające użytemu, domyślnemu rozmiarowi obrazu poddanemu detekcji w YOLOv8n, które wynosi 640x640px. Jest to również domyślna rozdzielczość wykorzystanego urządzenia, co może powodować szybszy pobór klatki z kamery.
Wpływ mogły mieć również czynniki zewnętrzne takie jak mniejsze zajęcie procesora przez inne programy działające w komputerze. Badania przeprowadzone w przyszłości mogłyby poruszać wpływ poszczególnych funkcji (np. detekcja i pobór klatki z kamery) na wydajność systemu. 
Testowanie oddzielnych części w ramach całego systemu jest jednak problematyczne ze względu na ograniczenie architektury wielowątkowej w Pythonie do wykonania jednego wątku w tym samym czasie, co może skutkować częstym przełączaniem się pomiędzy zadaniami, w efekcie zaburzając rzeczywisty czas wykonania indywidualnej funkcji.  

Zagęszczenie pokazuje pewien stopień powtarzalności. Mimo to, na tym etapie testów systemu nie można uznać za powtarzalny. Liczba przetestowanych klatek nie pozwala ocenić, czy istnieją pewne tendencje w zaobserwowanych odchyleniach -- np. w chwili zaistnienia pierwszej wartości uznanej za odchylenie, kolejne odczyty nie maleją do wartości bliższej średniej, lecz utrzymują się na poziomie odchylenia przez pewien czas. 
W celu dalszych testów można by więc parokrotnie zbadać mniejszą liczbę klatek (np. podzbiory otrzymanych wyników) i sprawdzić, czy odchylenia mają naturę całkowicie losową, bądź nie. Wykonanie takiego badania nie jest natomiast w pełni konieczne. Priorytetem systemu jest alarmowanie użytkownika z możliwie jak najmniejszym opóźnieniem, co eliminuje konieczność utrzymania stałego czasu wykonania. Biorąc to pod uwagę badaniu poddano by tylko odchylenia większe od średniej.

Sam fakt wystąpienia odchyleń nie definiuje, jak bardzo opóźnione są alarmy wysyłane przez system. Wykresy pokazały, że nawet w sytuacji wystąpienia odchylenia, wyniki dla kolejnych klatek dalej są w większości zagęszczone w okolicy średniej i ponadto nie wystąpiła uznana za znaczącą zmiana w stopniu zagęszczenia. Dodatkowo wykres na rysunku \ref{fig:czas-punktowy-all} pokazał również, iż zakres odchyleń można zaklasyfikować do tego samego zbioru dla wszystkich rozdzielczości.   
Podczas wykorzystania systemu przez użytkownika, jakościowo stwierdzono, iż opóźnienia w alarmowaniu są niezauważalne. 

Podsumowując, dla wykorzystanego wyposażenia sprzętowego, wykazano niezależność wyników od ustawionej rozdzielczości kamery. Ponadto stwierdzono, iż opóźnienia systemu wynikające z odchyleń od średniej nie wpływają na komfort użytkowania.