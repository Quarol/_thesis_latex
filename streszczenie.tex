\pdfbookmark[0]{Streszczenie}{streszczenie.1}
%\phantomsection
%\addcontentsline{toc}{chapter}{Streszczenie}
%%% Poni¿sze zosta³o niewykorzystane (tj. zrezygnowano z utworzenia nienumerowanego rozdzia³u na abstrakt)
%%%\begingroup
%%%\setlength\beforechapskip{48pt} % z jakiegoœ powodu by³a maleñka ró¿nica w po³o¿eniu nag³ówka rozdzia³u numerowanego i nienumerowanego
%%%\chapter*{\centering Abstrakt}
%%%\endgroup
%%%\label{sec:abstrakt}
%%%Lorem ipsum dolor sit amet eleifend et, congue arcu. Morbi tellus sit amet, massa. Vivamus est id risus. Sed sit amet, libero. Aenean ac ipsum. Mauris vel lectus. 
%%%
%%%Nam id nulla a adipiscing tortor, dictum ut, lobortis urna. Donec non dui. Cras tempus orci ipsum, molestie quis, lacinia varius nunc, rhoncus purus, consectetuer congue risus. 
%\mbox{}\vspace{2cm} % mo¿na przesun¹æ, w zale¿noœci od d³ugoœci streszczenia
\begin{abstract}
    Niniejsza praca opisuje wykonanie systemu do wykrywania obiektów w czasie rzeczywistym na obrazach z kamery. Celem systemu jest alarmowanie audio-wizualne w momencie pojawienia się wykrywanego obiektu w obiektywie kamery. W projekcie wykorzystano podstawową wersję modelu detekcji obiektów YOLOv8n. Dostępne klasy obiektów są określone przez zestaw danych, na którym przetrenowano w.w. model. Zakresem projektu jest implementacja systemu w postaci aplikacji desktopowej oraz testy szybkości wykonanego oprogramowania. Ponadto przeprowadzono badania skuteczności klasyfikacji YOLOv8n w zależności od wybranych warunków oświetleniowych w pomieszczeniu umieszczenia kamery oraz wartości progu ufności detektora.
\end{abstract}
\mykeywords

{
\selectlanguage{english}
\begin{abstract}
This paper describes the implementation of a system for real-time object detection from camera images. The purpose of the system is to trigger audio-visual alert when a detected object appears in the camera frame. The project uses a pretrained version of the YOLOv8n object detection model. The available classes of objects are determined by the dataset which was used to train the model on. The scope of the project is the implementation of the system in the form of a desktop application and conducting speed tests of the produced software. In addition, tests were carried out on the effectiveness of YOLOv8n classification depending on the selected lighting conditions in the camera placement area and the detector's confidence threshold value.
\end{abstract}
\mykeywords
}
