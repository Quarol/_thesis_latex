\section{Szczegóły implementacyjne}
\label{chap:szczegoly-implementacji}
W sekcji tej opisano implementację wielowątkowego podejścia architektonicznego. Przedstawiono argumenty za wybraniem tego typu architektury, rozwiazanie problemów przez nią generowanych oraz uzasadnienie innych działań zastosowanych w celu poprawy wydajności systemu.  

\subsection{Uwagi dotyczące architektury systemu}
Pierwsza iteracja oprogramowania była oparta na architekturze jednowątkowej. Sekwencyjnie wykonywano kolejne metody w nieskończonej pętli: pobór klatki z kamery, detekcja obiektów, alarmowanie dźwiękowe, wizualizacja obiektów oraz wyświetlenie klatki w wyświetlaczu graficznego interfejsu użytkownika.

Na tym etapie rozdziału warto zrozumieć działanie modułu graficznego użytego w klasie \emph{GUI} --- tkinter. 
Jest on podstawą i jedyną wykorzystywaną technologią w implementacji panelu sterowania. Moduł ten dostarcza graficzne komponenty takie jak suwak, wykorzystany do zmiany progu ufności; pola wyboru (ang. \emph{checkboxes}) wraz z paskiem przewijania, zastosowane w sekcji z wyborem wykrywanych klas obiektów; oraz menu z przyciskami, do wyboru źródła wideo. 

Aby móc użytkować interfejs na bazie modułu, wymagane jest wywołanie głównej pętli tkinter (ang .\emph{mainloop}). Pętla opiera się na architekturze wydarzeniowej (ang. \emph{event-driven architecture}) działającej w ramach jednego, głównego wątku aplikacji. Poprzez wydarzenie rozumie się zdefiniowaną w module integrację z komponentem np. kliknięcie na przycisk. Zadaniem pętli jest zareagowanie na wydarzenia w sposób zdefiniowany przez moduł (np. wypełnienie przycisku kolorem po kliknięciu) oraz wykonanie kodu, który programista podłączył pod dane wydarzenia (np. wywołanie metody ustawiającej źródło wideo). Przykład na bazie kodu źródłowego pokazano w listingu \ref{lst:tkinter-event-loop}.


\begin{lstlisting}[caption={Podłączenie metod pod wydarzenia oraz zainicjalizowanie pętli głównej w module tkinter.}, label={lst:tkinter-event-loop}]
def GUI:
    # Metoda tworzaca panel sterowania
    def _initialize_control_panel(self) -> None:
        self._root = tk.Tk() # Utworzenie aplikacji (okna) tkinter
        # Utworzenie menu do wyboru zrodla wideo:
        self._menubar = tk.Menu(master=self._root)
        self._root.config(menu=self._menubar)
        self._initialize_video_source_menu()
        # reszta kodu metody ...

    # Metoda tworzaca menu z przyciskami do wyboru zrodla wideo 
    def _initialize_video_source_menu(self) -> None:
        self._source_menu = tk.Menu(master=self._menubar, tearoff=0)
        self._selected_video_source_id = tk.IntVar()

        sources = self._frame_generator.get_available_sources()
        for source_name in sources:
            # Stworzenie pojedynczego przycisku:
            self._source_menu.add_radiobutton(
                label=source_name,
                variable=self._selected_video_source_id,
                value=sources[source_name],
                # Podlaczenia zaprogramowanej funkcji do ustawienia zrodla wideo do wywolania po nacisnieciu przycisku:
                command=lambda source_index=sources[source_name]: self._on_video_source_select(source_index) 
            )

        self._selected_video_source_id.set(NO_VIDEO)
        self._menubar.add_cascade(menu=self._source_menu, label='Video source')

    # Metoda uruchamiajaca GUI (inizjalizuje petle tkinter)
    def show(self) -> None:
        # ... kod poprzedzajacy
        self._root.mainloop()
        # reszta kodu... (zostanie wykonana po zakonczeniu petli)

\end{lstlisting}

Z perspektywy programistycznej, instrukcje następujące po inicjalizacji \emph{tkinter} zostaną wykonane dopiero po zakończeniu głównej pętli -- czyli zamknięciu okna aplikacji przez użytkownika.
Taka charakterystyka działania wymusza wywołanie sekcji logicznej aplikacji przed rozpoczęciem pętli, co jest problematyczne, ponieważ sekcja ta musi być częścią mechanizmu tej pętli, aby sekwencje w sekcji logicznej były powtarzane.
Rozwiązaniem oferowanym przez \emph{tkinter} jest funkcja \emph{after}.
Funkcja ta umożliwia wywołanie fragmentu kodu po zadanym interwale czasowym.
Przyjmuje dwa argumenty: liczbe całkowitą, która określa czas w milisekundach, po którym zostanie wykonany fragment kodu; funkcję, która jest wywoływanym po upływie interwału fragmentem kodu. 

Użycie funkcji \emph{after} nie blokuje wykonania kolejnych instrukcji w kodzie, dlatego może być bezpiecznie zastosowane w aplikacji opartej na \emph{tkinter}. Zaprogramowanym sposobem użycia jest wywołanie metody do komunikacji z sekcją logiczną \emph{\_update\_frame}. Istotą tego rowiązania jest zawarcie w kodzie tej metody najpierw obsługi sekcji logicznej oraz późniejsze wywołanie tej samej metody przy użyciu funkcji \emph{after}. Nieblokujący mechanizm \emph{after} unika nieskończonej rekurencji prowadzącej do pojawienia się wyjątków pamięciowych systemu operacyjnego. Mechanizm wyświetlania przy użyciu opisanych funkcji przedstawiono w listingu \ref{lst:tkinter-after}.

\begin{lstlisting}[caption={Wyświetlanie klatek obrazu w GUI przy pomocy modułu tkinter.}, label={lst:tkinter-after}]
AFTER_DELAY = 1 # Czas w milisekundach. Najmniejsze mozliwe ustawienie. 
class GUI:
    # Konstruktor klasy
    def __init__(self, communication_interface: App) -> None:
        # Ustawienie interfejsu do komunikacji z sekcja logiczna oprogramowania, w tym poboru klatek z kamery:
        self._communication_interface = communication_interface
        # reszta kodu ...

    # Metoda do uruchomienia GUI
    def show(self) -> None:
        self._root.mainloop() # inicjalizacja petli tkinter

    # Metoda uruchamiana w po ustawieniu kamery w panelu sterowania
    # Sluzy do rozpoczecia komunikacji z warstwa logiczna w celu poboru klatki do wyswietlenia  
    def _start_displaying(self) -> None:
        self._is_displaying = True
        cv.namedWindow('Display', cv.WINDOW_NORMAL) # Stworzenie okna wyswietlacza
        cv.moveWindow('Display', 0, 0)
        self._update_frame()

    # Metoda uruchamiana po zakonczeniu strumienia wideo
    # Sluzy zakonczeniu wyswietlania (zakonczenie poboru klatek)
    def _stop_displaying(self) -> None:
        self._is_displaying = False
        cv.destroyAllWindows() # Zniszczenie okna wyswietlacza

    def _update_frame(self) -> None:
        if not self._is_displaying:
            self._stop_displaying()
            return

        # Pobranie klatki (frame) oraz informacji o tym czy kamera dalej dziala (is_capture_on):
        is_capture_on, frame =  self._communication_interface.get_processed_frame()

        if is_capture_on:
            if frame is not None:
                self._show_frame(frame)
            # Jezeli kamera dziala, mechanizm kontunuowany:
            self._root.after(AFTER_DELAY, self._update_frame)
        else:
            self._frame_generator.set_video_source(NO_VIDEO)
            self._selected_video_source_id.set(NO_VIDEO)
            self._stop_displaying()

    # Metoda wywolywana do wyswietlania klatki obrazu
    def _show_frame(self, frame: MatLike) -> None:
        frame = cv.cvtColor(frame, cv.COLOR_BGR2RGB)
        cv.imshow('Display', frame) # Wyswietlenie klatki
\end{lstlisting}

Użycie interfejsu do pobrania klatki (\emph{self.\_communication\_interface.get\_processed\_frame()}) pozwoliło oddzielić sposób pozyskiwania klatki od architektury komunikacyjnej między sekcją GUI a sekcją logiczną oraz zachować taki sam kod klasy \emph{GUI} dla następnych iteracji systemu. W przypadku iteracji jednowątkowej użycie metody interfejsu wywoływało następującą sekwencję kroków: pobór klatki z kamery; zmniejszenie rozmiaru klatki, tak aby zmieściła się na ekranie komputera; detekcja obiektów; alarmowanie dźwiękowe oraz wizualizacja prostokątów ograniczających w przypadku wykrycia obiektów; zwrócenie przetworzonej klatki obrazu.

Długa sekwencja zadań oraz czasochłonność niektórych operacji (np. detekcji obiektów) blokowały odświeżanie oraz dostęp do funkcji komponentów GUI, czyniąc go oraz cały system nieresponsywnym i powolnym. W celu poprawy optymalizacji nie zdecydowano się zmienić ustawionego minimalnego czasu, po którym następują kolejne wykonania \emph{\_update\_frame}, ponieważ wiązałoby się to z ryzykiem utraty większej liczby pobieranych klatek -- kluczowych dla alarmowania użytkownika -- pomimo poprawy responsywności interfejsu graficznego.

Pierwszym podjętym krokiem w celu poprawy wersji jednowątkowej było przeniesienie wykonania inferencji (detekcji) z CPU na GPU. Parametr do tego użyty był opisany w rodziale \ref{chap:wprowadzenie-yolo_interjes}. Poprawiono tym znacznie wydajność operacji uzyskania finalnej klatki, aczkolwiek GUI nadal borykało się z problemem responsywności. Dlatego postanowiono zmienić architekturę na wielowątkową. Podejściem takim osiągniąto zamierzone rezultaty -- GUI osiągneło zadowalający poziom responsywności przy zachowaniu satysfakcjonującej wydajności (opisanej w rozdziale \ref{chap:test-szybkosci}, testów szybkości systemu).







\subsection{Szczegóły implementacyjne wątków}
Wątki zdefiniowano według opisu architektury w rozdziale \ref{chap:architektura}. Wspomniano w nich o zastosowaniu mechanizmów synchronizacji dostępu do buforów. W tej sekcji tego dokumentu przedstawiono użyte mechanizmy oraz omówiono fragmenty kodu objętych wątkami.

Zaczynając od wątku do alarmowania: w celu zapobiegnięcia uruchomieniu alarmu w momencie, gdy poprzedni się nie zakończył (czyli gdy plik dźwiękowy nie skończył odtwarzania), zastosowano rozwiązanie z biblioteki do obsługi wątków oraz dostarczającej mechanizmy synchronizacji wątków -- \emph{threading} -- \emph{Event}. Użyto trzech funkcji obiektu Event: \emph{set}, \emph{clear} oraz \emph{is\_set}. \emph{set} ustawia wewnętrzną flagę w obiekcie Event, natomiast \emph{clear} -- czyści ją. Funkcja \emph{is\_set} sprawdza czy flaga została ustawiona. W metodzie bezpośrednio uruchamiającej alarm (\emph{\_play\_alert}) najpierw ustawiono flagę, a następnie uruchomiono alarm. Po zakończeniu alarmu flaga jest czyszczona, a metoda kończy swoje działanie. Metoda wywołująca \emph{\_play\_alert} (\emph{play\_audio\_alert}) sprawdza, czy flaga została ustawiona. Jeżeli tak, kończy ona działanie. Jeżeli nie, wywołuje ona \emph{\_play\_alert} w oddzielnym wątku. Do przydzielenia metody do oddzielnego wątku użyto obiektu klasy \emph{ThreadPoolExecutor} z biblioteki \emph{concurrent.futures}, służącego do automatycznego zarządzania wątkami. Z kolei do odtworzenia pliku dźwiękowego wykorzystano metodę \emph{playsound}.
Kod ilustrujący uruchamianie alarmu z wykorzystaniem opisanej powyżej metodologii znajduje się w listingu \ref{lst:alarm-thread}.

\begin{lstlisting}[caption={Kod uruchamiający alarm dźwiękowy oraz zapobiegający wielu symultaicznym uruchomieniom.}, label={lst:alarm-thread}]
# playsound - umozliwia odtworzenie pliku dzwiekowego:
from playsound import playsound 
# Biblioteka do obslugi wielowatkowosci i mechanizmow synchornizacji:
import threading
# ThreadPoolExecutor - zarzadza automatycznie watkami:
from concurrent.futures import ThreadPoolExecutor

ALERT_SOUND = 'assets/alert.wav' # sciezka do pliku dzwiekowego
class App:
    # Konstruktor App:
    def __init__(self) -> None:
        # ... reszta kodu
        self._play_alert_executor = ThreadPoolExecutor(max_workers=1)
        self._alert_event = threading.Event()
        self._alert_event.clear()    
        # reszta kodu ...

    # Metoda wywolana przez VideoProcessingEngine ...
    # ... w celu odtworzenia alarmu (pliku dzwiekowego):
    def play_audio_alert(self) -> None:
        # Sprawdzenie czy alarm sie zakonczyl:
        if not self._alert_event.is_set():
            # Jezeli tak, otworz plik dzwiekowy w oddzielnym watku: 
            self._play_alert_executor.submit(self._play_alert)
        # Jezeli nie, alarm nie jest wlaczany

    # Metoda odtwarzajaca plik dzwiekowy:
    def _play_alert(self) -> None:
        # Ustawienie eventu, zapobieganie odwtwarzania ...
        # ... wiecej niz raz w tym samym czasie: 
        self._alert_event.set()
        # Odtworzenie alarmu:
        playsound(ALERT_SOUND)
        # Wyczyszcenie eventu, przywrocenie mozliwosci ... 
        # ... odtworzenia alarmu:
        self._alert_event.clear()
\end{lstlisting}


Przechodząc do wątków podmodułów opisanych w rodziale \ref{chap:architektura}, omówiony zostanie mechanizm bufora pośredniego pomiędzy podmodułami oraz mechanizm bufora wyjściowego pomiędzy podmodułem drugim a periodycznym poborem klatki przez graficzny interfejs użytkownika. Omawiane mechanizmy są przedstawione w kodzie w listingu \ref{lst:engine-1}. Metoda \emph{\_capture\_frames} realizuje kroki podmodułu pierwszwego, zaś \emph{\_process\_frames} podmodułu drugiego.
%najpierw zostanie opisany mechanizm synchronizacji dostępu do klatek, po czym przedstawiony zostane mechanizm reagujący na zmiane bądź usunięcie źródła wideo przez użytkownika.

W przypadku bufora pośredniego \emph{\_buffer} zastosowano mechanizm synchronizacji przy użyciu \emph{Lock} (mutex) oraz zmiennej warunkowej związanej z tym mutexem (\emph{\_buffer\_lock}). 
Strona pobierająca klatkę z bufora dokonuje próby uzyskania mutexa poprzez zmienną warunkową. Strona pobierająca klatkę z bufora próbuje uzyskać mutex za pomocą zmiennej warunkowej. Następnie, jeśli bufor jest pusty, wątek zostaje uśpiony na tej zmiennej warunkowej, aby nie marnować zasobów obliczeniowych. Jeśli bufor nie jest pusty lub wątek zostaje wybudzony, pobierana jest klatka, a bufor jest czyszczony. 

Po stronie wysyłającej klatkę do bufora, na początku następuje próba uzyskania mutexa. Jest to ten sam mutex, co po stronie pobierającej, co eliminuje problem wyścigu. Po uzyskaniu mutexa klatka jest umieszczana w buforze, a następnie, przy użyciu funkcji \emph{notify}, wybudzany jest wątek podmodułu drugiego. 

Zdecydowano się na wykorzystanie bufora przechowującego tylko jedną klatkę, ponieważ w systemie priorytetem jest przetwarzanie najnowszej klatki z minimalnym opóźnieniem względem kamery, co ma kluczowe znaczenie dla szybkości alarmowania -- głównego celu tego systemu.
W hipotetycznym przypadku użycia kolejki o rozmiarze większym niż jeden, gdy podmoduł pierwszy (o szybkości głównie zależnej od FPS kamery) jest szybszy od podmodułu drugiego, kolejka co pewien czas zapełniałaby się.
Taka sytuacja wymagałaby zatrzymania podmodułu pierwszego, co wiązałoby się z niedostarczaniem najnowszych klatek do bufora.
Kolejną potencjalną wadą mechanizmu kolejkowania jest jego działanie oparte na zasadzie FIFO. 
W przypadku wypełnionej kolejki, zamiast przetwarzania najnowszej klatki, pobierane byłyby starsze klatki niż najnowsza dostępna. O ile nie stanowiłoby to problemu, gdyby detekcja obiektów była wystarczająco szybka, a czas detekcji względnie powtarzalny, badania opisane w rozdziale \ref{chap:badania-skutecznosci} pokazują, iż istnieją znaczące odchylenia od średniego czasu detekcji. W takiej sytuacji mógłby być wymagany dodatkowy mechanizm, który na przykład czyściłby kolejkę i pobierał z niej najnowszą klatkę, jeżeli stwierdzono, że system ma zbyt duże opóźnienie względem kamery. Jednak z racji niepodanych wymagań sprzętowych, które system ma spełniać, zdecydowano się pozostać przy aktualnym, prostszym mechanizmie synchronizacji.

Mechanizm synchronizacji użyty pomiędzy GUI a drugim podmodułem jest podobny, z tą różnicą, że nie wykorzystano zmiennej warunkowej. Moduły uzyskują dostęp do bufora po uprzednim uzyskaniu dostępu do mutexa. 


\begin{lstlisting}[caption={Kod modułu silnika przetwarzania wideo z zawartymi mechanizami synchronizacji}, label={lst:engine-1}]
class VideoProcessingEngine:
    def __init__(self, video_capture: VideoCapture, image_processor: ImageProcessor, audio_alarm: App) -> None:
        self._frame_buffer = None # Bufor posredni
        self._processed_frame_buffer = None # Bufor wyjsciowy klatki
        self._is_capture_on = False # Okresla czy kamera dziala

        self._buffer_lock = threading.Lock()
        self._buffer_not_empty = threading.Condition(self._buffer_lock)

        # Warunek petli podtrzymujacy istnienie watkow 
        # Jest ustawiany na False kiedy aplikacja zostaje wylaczona
        self._continue_thread_loop = True

        # Watek do podmodulu drugiego:
        self._processing_thread = threading.Thread(target=self._process_frames, daemon=True)
        # Watek do podmodulu pierwszego:
        self._capture_thread = threading.Thread(target=self._capture_frames, daemon=True)
        

    # Metoda sprawdzajaca czy bufor posredni jest pusty:
    def _is_buffer_empty(self) -> bool:
        return self._frame_buffer is None 
    
    # Metoda wstawiajaca klatke do bufora posredniego:
    def _set_buffer(self, frame: MatLike) -> None:
        self._frame_buffer = frame
    
    # Metoda pobierajaca klatke z bufora posredniego ... 
    # oraz go czyszczaca:
    def _fetch_buffer(self) -> MatLike|None:
        frame = self._frame_buffer
        self._frame_buffer = None
        return frame

    # Metoda wstawiajaca przygotowana klatke ... 
    # ... do bufora wyjsciowego:
    def _set_processed_frame_buffer(self, frame: MatLike) -> None:
        self._processed_frame_buffer = frame

    # Metoda pobierajaca klatke z bufora wyjsciowego ... 
    # oraz go czyszczaca:
    def _fetch_processed_frame_buffer(self) -> MatLike:
        frame = self._processed_frame_buffer
        self._processed_frame_buffer = None

        return frame

    # Metoda uruchamiajaca modul (wywolana przez App)
    def run(self) -> None:
        self._processing_thread.start()
        self._capture_thread.start()

    # Metoda implementujaca kroki podmodulu pierwszego:
    def _capture_frames(self) -> None:
        while self._continue_thread_loop:
            if not self._capture_event.is_set():
                # Czekanie az kamera zostanie ustawiona:
                self._capture_event.wait()
                # Jezeli nastapilo wylaczenie aplikacji, zakoncz petle: 
                if not self._continue_thread_loop:
                    return

            # ... 
            # Kroki podmodulu pierwszego
            # ...
            
            # Mechanizm bufora posredniego po stronie wysylajacej klatke
            with self._buffer_lock:
                self._set_buffer(frame)
                self._buffer_not_empty.notify()

    # Metoda implementujaca kroki podmodulu drugiego:
    def _process_frames(self) -> None:
        while self._continue_thread_loop:
            # ... reszta kodu

            # Mechanizm bufora posredniego po stronie pobierajacej klatke
            with self._buffer_not_empty:
                while self._is_buffer_empty():
                    self._buffer_not_empty.wait()
                    # Jezeli nastapilo wylaczenie aplikacji, zakoncz petle:
                    if not self._continue_thread_loop:  
                        return

                frame = self._fetch_buffer() 

            # ... 
            # Kroki podmodulu pierwszego
            # ...

            # Mechanizm bufora wyjsciowego po stronie wysylajacej klatke
            with self._frame_set_lock:
                self._set_processed_frame_buffer(frame)

    # Metoda wywolywana periodycznie przez GUI ... 
    # ... za posrednictwem App ... 
    # w celu poboru gotowej klatki do wyswietlenia ... 
    # oraz informacji czy kamera dalej dziala
    def get_processed_frame(self) -> Tuple[bool, Optional[MatLike]]:
        # Mechanizm bufora wyjsciowego po stronie pobierajacej klatke
        with self._frame_set_lock:
            is_capture_on = self._is_capture_on
        frame = self._fetch_processed_frame_buffer()
        
        return is_capture_on, frame
\end{lstlisting}