\section{Wymagania funkcjonalne i założenia systemowe}
\label{chap:wymagania-funcjonalne}
Projektowany system składa się z różnych funkcji zsynchronizowanych ze sobą. Będzie on realizowany w postaci oprogramowania. 
Oprogramowanie będzie stworzone w formie aplikacji desktopowej. Aplikacja ta musi posiadać graficzny interfejs użytkownika (GUI), składający się z panelu sterowania oraz wyświetlacza. 
Oprogramowanie ma obsługiwać jedną z dostępnych (niewykorzystywanych przez inne procesy) kamer podłączonych do komputera. Lista dostępnych kamer jest określana jednorazowo przed pojawieniem się GUI na ekranie. Dlatego w przypadku chęci użycia kamery podłączonej w trakcie działania programu, należy zamknąć aplikację (panel sterowania GUI) i uruchomić ją ponownie.

Na bazie powyższych założeń kolejne wymagania funkcjonalne można podzielić na dwa bloki: pierwszy (logiczny) i drugi (wpływający na stan GUI). Na podstawie pierwszego bloku system realizuje następujące funkcje:
\begin{itemize}
    \item Pobiera klatki obrazu z kamery.
    \item Zmniejsza rozmiar klatki, jeżeli jest on większy niż rozdzielczość ekranu monitora komputera.
    \item Wykrywa obiekty na podanej klatce obrazu.
    \item Przygotowuje klatkę obrazu do wyświetlenia -- wizualizuje lokalizację obiektów na podstawie wyników detekcji, jeżeli znajdują się w nich wykrycia.
    \item Alarmuje dźwiękowo użytkownika, jeżeli obiekty zostały wykryte.
    \item Synchronizuje wyżej wypunktowane funkcje.
\end{itemize}

Na podstawie drugiego bloku system:
\begin{itemize}
        \item Wyświetla przygotowane klatki obrazu w wyświetlaczu GUI.
        \item Umożliwia użytkownikowi wyłączenie źródła wideo (wyłączenie wyświetlacza) w panelu sterowania GUI.
        \item Umożliwia użytkownikowi wybór kamery jako źródło wideo z dostępnych kamer wejściowych w panelu sterowania GUI.
        \item Umożliwia użytkownikowi ustawienie progu ufności detekcji (opisanego w rozdziale \ref{chap:wprowadzenie-yolo_interjes}) obiektów w panelu sterowania GUI.
        \item Umożliwia użytkownikowi ustawienie wykrywanych klas obiektów z określonego zbioru klas w panelu sterowania GUI.
        \item Generuje listę dostępnych kamer przed wyświetleniem GUI.
\end{itemize}

System musi zorganizować funkcje w moduły i odpowiednio je synchronizować. Założoną architekturą systemu jest architektura wielowątkowa. 