\section{Wymagania funkcjonalne i założenia systemowe}
Projektowany system będzie służył synchronizacji różnych wymagań funkcjonalnych opisanych w niniejszym rozdziale. Będzie on wykonany w postaci oprogramowania. 
Oprogramowanie będzie wytworzone w formie aplikacji desktopowej, dostępnej na system operacyjny Windows 11. Aplikacja ta musi posiadać graficzny interfejs użytkownika (GUI) składający się z panelu sterującego oraz wyświetlacza. 
Oprogramowanie ma obługiwać jedną z dostępnych (niewykorzystywanych przez inne procesy) kamer podłączonych do komputera. Lista dostępnych kamer jest określana jednorazowo przed pojawieniem się GUI na ekranie. Dlatego w przypadku chęci użycia kamery podłączonej w trakcie wykonywania programu należy zamknąć aplikację (panel sterowania GUI) i uruchomić ją ponownie.

Na podstawie powyższych założeń, kolejne wymagania fukncjonalne można podzielić na dwa bloki: pierwszy (logiczny), drugi (wpływający na stan GUI). Na bazie pierwszego bloku system realizuje następujące funkcje:
\begin{itemize}
    \item Pobiera klatki obrazu z kamery.
    \item Zmniejsza rozmiar klatki, jeżeli jest on większy niż ekran monitora komputera.
    \item Wykrywa obiekty na podanej klatce obrazu.
    \item Przygotowuję klatkę obrazu do wyświetlenia -- wizuallizuje lokalizację obiektów bazując na wynikach detekcji, jeżeli znajdują się w nich wykrycia.
    \item Alarmuje dźwiękowo użytkownika, jeżeli obiekty zostały wykryte.
    \item Synchronizuje wyżej wypunktowane funkcje.
\end{itemize}

Na bazie drugiego bloku system:
\begin{itemize}
        \item Wyświetla przygotowane klatki obrazu w wyświetlaczu GUI.
        \item Umożliwia użytkownikowi wyłączenia źródła wideo (wyłączenia wyświetlacza) w panelu sterowania GUI.
        \item Umożliwia użytkownikowi ustawienie progu ufności detekcji (próg ufności opisany w rozdziale \ref{chap:wprowadzenie-yolo_interjes}) obiektów w panelu sterowania GUI.
        \item Umożliwia użytkownikowi ustawienie wykrywanych klas obiektów ze zdefiniowanego zbioru klas w panelu sterowania GUI.
        \item Umożliwia użytkownikowi wybór kamery jako źródło wideo z dostępnych kamer wejściowych w panelu sterowania GUI.
        \item Generuje listę dostępnych kamer przed wyświetleniem GUI.
\end{itemize}

System musi zorganizować funkcje w odpowiednie moduły i odpowiednio je synchronizować. Założoną architekturą systemu jest architektura wielowątkowa. 