\section{Architektura systemu}
\label{chap:architektura}
Najważniejsze elementy systemu można podzielić na dwa główne moduły: silnik przetwarzania wideo oraz interfejs użytkownika. 

Silnik przetwarzania wykonuje następujące zadania:
\begin{enumerate}
    \item Pobiera klatkę obrazu z kamery.
    \item Zmniejsza rozmiar klatki, jeżeli jest on większy niż rozdzielczość ekranu monitora komputera
    \item Wykrywa obiekty na podanej klatce obrazu.
    \item Przygotowuje klatkę obrazu do wyświetlenia --wizualizuje lokalizację obiektów na podstawie wyników detekcji, jeżeli znajdują się w nich wykrycia.
    \item Alarmuje dźwiękowo użytkownika, jeżeli obiekty zostały wykryte.
    \item Przekazuje przygotowaną klatkę na wyjście modułu. 
\end{enumerate}

Operacja detekcji jest czasochłonna, jak również pobór klatki, ze względu na ograniczenia sprzętowe kamery (FPS). Dlatego zamiast sekwencyjnej realizacji zadań silnika przetwarzania, postanowiono podzielić go na dwa podmoduły i zrównoleglić ich wykonanie, tworząc wątek dla każdego z nich. Zadania zawarte w każdym podmodule są wykonywane sekwencyjnie.

Pierwszy podmoduł realizuje w kolejności zadania nr 1 i nr 2, zaś moduł drugi -- w kolejności zadania od nr 3 do 6 włącznie. Przekazanie klatki pomiędzy wątkami (modułami) umożliwiono poprzez zastosowanie bufora (nazwanego buforem pośrednim) wraz z mechanizmami synchronizacji dostępu do niego (dalej opisanymi w szczegółach implementacyjncyh w rozdziale \ref{chap:szczegoly-implementacji}). Podmoduł pierwszy przekazuje klatkę do bufora przejściowego, natomiast podmoduł drugi czeka aż klatka ta zostanie przekazana po czym pobiera ją i czyści bufor przejściowy. 

Oprócz ustanowienia podmodułu drugiego oddzielnym wątkiem, podmoduł ten wywołuje wykonanie alarmu dźwiękowego (krok nr 4) w dodatkowym wątku. Bez użycia dodatkowego wątku dalsze kroki w podmodule zostałyby wykonane dopiero po zakończeniu alarmu -- po zakończeniu pliku dźwiękowego służącego jako alarm. Sekwencja podmodułu w większości przypadków wykonuje się szybciej niż odtworzenie całego pliku co w połączeniu z dodatkowym wątkiem może powodować nieprzyjemne dla użytkownika nałożenie się alarmów w tym samym czasie. Zastosowano więc mechanizm (opisany w rozdziale \ref{chap:szczegoly-implementacji}), który pozwala uniknąć wywołania alarmu, jeżeli poprzedni nie został zakończony.

Przechodząc do modułu interfejsu użytkownika, wykonuje on następujące funkcje systemu:
\begin{enumerate}
    \item Wyświetla przygotowane klatki obrazu w wyświetlaczu GUI.
    \item Umożliwia użytkownikowi wyłączenie źródła wideo (wyłączenie wyświetlacza) w panelu sterowania GUI.
    \item Umożliwia użytkownikowi wybór kamery jako źródło wideo z dostępnych kamer wejściowych w panelu sterowania GUI.
    \item Umożliwia użytkownikowi ustawienie progu ufności detekcji (próg ufności opisany w rozdziale \ref{chap:wprowadzenie-yolo_interjes}) obiektów w panelu sterowania GUI.
    \item Umożliwia użytkownikowi ustawienie wykrywanych klas obiektów z określonego zbioru klas w panelu sterowania GUI.
\end{enumerate}
Ważną informacją na temat funkcji od nr 2 do nr 4 są okoliczości ich wywołania. Zainicjowane zostają jako efekt integracji użytkownika z panelem sterowania graficznego interfejsu użytkownika. Temat użycia GUI opisano w oddzielnym rozdziale \ref{chap:uzytkownik-gui}. 

Moduł interfejsu użytkownika również działa w oddzielnym wątku. Aby krok nr 1 był możliwy, wymagana jest komunikacja z modułem silnika (dokładniej podmodułem drugim).  
Należało, tak jak w przypadku podmodułów silnika przetwarzania wideo, zaimplementować sposób przekazywania klatki. Również użyto do tego bufora (nazwanego buforem wyjściowym) z mechanizmem synchronizacji dostępu. 
Również użyto do tego bufora (nazwanego buforem wyjściowym) z mechanizmem synchronizacji dostępu. W tym przypadku oprócz klatki przesyłana jest również informacja, czy kamera dalej działa, co może być przydatną informacją w przypadku odłączenia kamery od komputera.
W przeciweństwie do podmodułu drugiego silnika, strona pobierająca dane (moduł interfejsu) nie czeka, aż bufor jest pełny. Próba poboru klatki jest realizowana periodycznie. W sytuacji braku klatki periodyczny pobór jest kontuuowany. W sytacji kiedy przekazano informacje o niedziałaniu kamery, periodyczny pobór zostaje zakończony.

Periodyczny pobór jest wyłączony na starcie aplikacji i wywoływany  dopiero poprzez ustawienie kamery funkcją nr 2. Pobór jest wyłączony również w przypadku wykonania funkcji nr 3.

Oprócz wpływu na moduł interfejsu, funkcje nr 2 i 3 ingerują w zachowanie modułu silnika, więc komunikują się z nim. Domyślnie, po włączeniu aplikacji, wątki dla obu podmodułów silnika są utworzone lecz ich sekwencje nie rozpoczynają się dopóki nie zostanie wysłany sygnał. Sygnał jest informacją o tym, iż kamera została ustawiona i działa. Sygnał ten wysyłany jest za pomocą funkcji nr 3. W przeciwieństwie do nr 3, funkcja nr 2 usuwa źródło wideo, co wstrzymuje dalsze powtórzenia obu sekwencji, tym samym oczekując na wznownienie wywołaniem funkcji nr 3.

Funkcje nr 4 i 5 ustawiają parametry modułu zawierającego model YOLOv8n, który jest konieczny do przeprowadzenie kroku nr 3 silnika przetwarzania wideo. 
W module tym wykorzystywany jest interfejs opisany w rozdziale \ref{chap:wprowadzenie-yolo_interjes}. 
Funkcja nr 4 ustawia parametr \emph{conf}, zaś nr 5 -- parametr \emph{classes}. 
Funkcje te zmieniają jedynie te parametry, nie ingerując w działanie modułu silnika przetwarzania wideo, co pozwala na kontynuowanie całego procesu kończącego się wyświetleniem klatki na wyświetlaczu GUI.
Domyślne parametry detektora to 0.5 dla progu ufności oraz klasa \emph{człowiek} jako zbiór klas obiektów poddanych detekcji.

Podsumowując, aplikacja wykorzystuje cztery wątki. Wyróżnić w niej można dwa główne moduły: moduł graficznego interfejsu użytkownika oraz moduł silnika przetwarzania wideo. Poprzez integracje użytkownika z panelem sterowania GUI, moduł interfejsu kontroluje kluczowowe sekwencje, które dzięki skomunikowaniu realizują cel niniejszego systemu -- alarmowanie audio-wizualne.  