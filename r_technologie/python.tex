Python jest to wysokopoziomowy, interpretowany język programowania. Jest on szeroko stosowany w dziedzinie sztucznej inteligencji, w tym wizji komputerowej. Za przykład może posłużyć opisany w rozdziale \ref{chap:wprowadzenie-yolo_interjes} interfejs. Pomimo szerokiego użycia w dziedzinie, istnieją optymalizacyjne problemy, które Python może generować. Ninejszy system jest opart na architekturze wielowątkowej. Tzw. GIL (ang. \emph{Global Interpreter Lock}) dopuscza możliwość by tylko jeden wątek programu mógł działać w tym samym czasie. Wprowadza to dodatkowy narzut mogący spowalniać aplikację. Istnieje podejście umożliwiające prawdziwe wykonanie współbierzne, ale jest one oparte na wieloprocesowości, co wiąże się z innym rodzajem narzutu, bardziej skomplikowana komunikacją między procesami w porównaniu do wątków. Ostatecznie, szerokie zastosowanie języka w dziedzinie oraz wspomniany intefejs programistyczny zadecydowały o użyciu tego własnie języka. Informacje przedstawione w tym rozdziale wraz ze specyfikacją techniczną języka można znaleść w oficjalnej dokumentacji \cite{Python_docs}. 