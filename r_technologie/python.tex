Python jest to wysokopoziomowy, interpretowany język programowania. Jest on szeroko stosowany w dziedzinie sztucznej inteligencji, w tym wizji komputerowej. Za przykład może posłużyć opisany w rozdziale \ref{chap:wprowadzenie-yolo_interjes} interfejs. 

Implementacja języka generuje jednak problemy wydajnościowe. Ninejszy system jest oparty na architekturze wielowątkowej. Tzw. GIL (ang. \emph{Global Interpreter Lock}) dopuszcza możliwość by tylko jeden wątek programu mógł działać w tym samym czasie. Wprowadza to dodatkowy narzut mogący spowalniać aplikację. Istnieje podejście umożliwiające prawdziwe wykonanie współbieżne, ale jest one oparte na wieloprocesowości, co wiąże się z innym rodzajem narzutu, bardziej skomplikowaną komunikacją między procesami w porównaniu do wątków. Ostatecznie, szerokie zastosowanie języka w dziedzinie oraz wspomniany intefejs programistyczny zadecydowały o użyciu tego własnie języka. Informacje przedstawione w tym rozdziale wraz ze specyfikacją techniczną Pythona można znaleść w oficjalnej dokumentacji \cite{Python_docs}. 